\section{Podsumowanie}
\label{sec:podsumowanie}

W~ramach projektu, przeprowadzona została analiza problemu automatycznego grupowania zawodników z~gry FIFA~19, według ich nominalnej pozycji. Przetestowano działanie sześciu różnych algorytmów grupowania, przy zastosowaniu czterech różnych miar niepodobieństwa. Dodatkowo, jakość grupowania została zbadana przy użyciu siedmiu różnych wskaźników.

Uzyskane wyniki są dalekie od zadowalających. Dla pierwszego podejścia do grupowania, korzystającego ze~zbioru $20$ atrybutów, obliczone wskaźniki jakości wskazują na niską jakość wewnętrzną uzyskanych grup. W~ogólności, uzyskane etykietowania są słabo określone, rzadkie oraz nieseparowalne. 

Dla grupowań korzystających z~klasycznych metryk niepodobieństwa, dzielących zbiór na dwie grupy, wartości miar zewnętrznych osiągają maksymalne wartości. Algorytmy skutecznie podzieliły zawodników na tych grających w~polu oraz tych grających na bramce. Stało się tak, ponieważ pomiędzy w~analizowanej przestrzeni atrybutów znajdował się jawny atrybut odpowiadający za grę na bramce (którego wartości były wysokie dla bramkarzy a niskie dla reszty zawodników).

Pozostałe rodzaje grupowań ze względu na ogólną i~szczegółową pozycję nie odniosły już takiego sukcesu. Wynika to z~braku jawnego atrybutu opisującego jak dobry zawodnik jest na danej pozycji oraz z~faktu że na jednej pozycji grają zarówno zawodnicy o wysokich i~niskich wartościach atrybutu. W~przypadku grupowania ze względu na pozycję, bardziej interesujące są wzorce jak poszczególne atrybuty mają się do siebie a~nie bezwzględne różnice pomiędzy poszczególnymi wartościami. 

Z~tego powodu, duże nadzieje wiązaliśmy z~grupowaniem opartym o~miarę korelacyjną. Niestety, wyniki przedstawione w~sekcji~\ref{sec:wyniki} ukazały niewątpliwą porażkę tego podejścia. Uzyskane wyniki zdecydowanie odstawały od wyników uzyskanych dla pozostałych miar. Problemy tej miary mogą być związane z~za dużą liczbą atrybutów, które wprowadzały niepotrzebny szum informacyjny.

Dużą poprawę przyniosło ograniczenie przestrzeni atrybutów z~dwudziestu do czterech. Dla tak przygotowanego zbioru, wyniki grupowania zdecydowanie poprawiły się. Dla miar klasycznych, kształt zależności wskaźnika średniej szerokości sylwetki nie zmienił się, jednak wartość bezwzględna wzrosła o~średnio $\num{0.2}$. Dla miary korelacyjnej, zależność diametralnie zmieniła charakter i~utrzymywała stały poziom niezależnie od liczby grup, na względnie wysokim poziomie. 

Grupy uzyskane w~wyniku grupowania danych na ograniczonej przestrzeni atrybutów charakteryzowały się wyższą jakością wewnętrzną: były lepiej zdefiniowane, gęstsze oraz bardziej separowalne. Jakość zewnętrzna grupowania uległa niewielkiej poprawie, szczególnie w~przypadku grupowania \emph{ze względu na ogólną pozycję}. Okazuje się że z~racji mnogości klas, grupowanie ze względu na szczegółową pozycję jest zadaniem wyjątkowo trudnym.

Interesującym elementem naszych badań było przetestowanie algorytmów grupowania rozmytego. Badany zbiór danych opisujących piłkarzy, w~naszej opinii idealnie nadaje się do grupowania rozmytego, ponieważ pozycja na boisku jest w~ogólności zmienną rozmytą. W~przypadku grupowania na pełnym zbiorze atrybutów, wartości rozmytego wskaźnika średniej szerokości sylwetki pozostawiały sporo do życzenia, informując o~niskiej jakości wewnętrznej grupowania. Zmniejszenie liczby atrybutów pozwoliło na zwiększenie wartości wskaźnika, zarówno dla algorytmu FANNY, jak i~algorytmu rozmytych k-średnich. 

Zarówno dla zbioru pełnego, jak i~ograniczonego, najlepsze rezultaty grupowania rozmytego udało nam się uzyskać dla miary korelacyjnej, z~którą wiązaliśmy największe nadzieje na potwierdzenie pierwotnej hipotezy. Najwyższe wyniki wskaźników jakości udało uzyskać się dla wysokich wartości $k$, co może świadczyć że przyjęty podział na szczegółową pozycję jest zbyt wąski. Dokładna analiza uzyskanych grup, powinna umożliwić odkrycie bardziej szczegółowych pozycji, omówionych w~\ref{subsec:atrybuty}.

Zastosowanie grupowania gęstościowego nie pozwoliło nam uzyskać zadowalających wyników. Optymalne rozwiązanie, cechowało się wysoką jakością wewnętrzną, jednak klasyfikowało większość zawodników jako szum, co w~naszej opinii jest zachowaniem błędnym. Mimo to, algorytm DBSCAN jest bardzo interesujący i~nie należy ignorować jego istnienia, uzyskane wyniki mogą być wykorzystane w~innych zadaniach obejmujących analizę zawodników w~profesjonalnych drużynach piłkarskich.

Uzyskane wyniki nie pozwalają nam jednoznacznie potwierdzić lub odrzucić hipotezy postawionej na samym początku. Wyniki uzyskane za pomocą grupowania opartego o~miary klasycznej pokazuje że możliwym jest skuteczne oddzielenie zawodników z~pola od bramkarzy. Wyniki uzyskane przez miarę korelacyjną wskazują że możliwym jest grupowanie po wzorcach, jednak otrzymane wyniki są zdecydowanie za słabe, aby uznać że udało się skutecznie pogrupować zawodników po ich optymalnej pozycji. 

Z~realizacji projektu wyciągnęliśmy dwie nauczki na przyszłość. Pierwszą z~nich jest fakt że \emph{im więcej, tym wcale nie jest lepiej}. Zbyt szeroka przestrzeń atrybutów może wprowadzać za duży szum, co może wpływać na uzyskaną jakość grupowania. Drugą nauczką na przyszłość jest możliwość wykorzystywania miar niepodobieństwa, innych niż klasyczne. W~zadaniach gdzie interesują nas wzorce a~nie wartości bezwzględne, lepiej jest korzystać z~miar opartych o~korelację atrybutów lub innych miar kosinusowych.