\documentclass{article}
\pdfpagewidth=8.5in
\pdfpageheight=11in

\usepackage{../ZUMreport}
\usepackage{times}
\usepackage{url}
\usepackage{xcolor}
\usepackage{polski}
\usepackage[polish]{babel}
\usepackage[utf8]{inputenc}
\usepackage[T1]{fontenc}
\usepackage[utf8]{luainputenc}
\usepackage[hidelinks]{hyperref}
\usepackage[utf8]{inputenc}
\usepackage{caption}
\usepackage{indentfirst}
\usepackage{graphicx}
\usepackage{amsmath}
\usepackage{float}
\usepackage{siunitx}
\usepackage{booktabs}
\usepackage{subfig}
\usepackage{pgfplots}
\usepackage{paracol}
\usepackage{gensymb}
\pgfplotsset{compat=1.17}

\urlstyle{same}
	
\title{Zaawansowane uczenie maszynowe\\ Sprawozdanie końcowe z projektu}

\author{
Robert Wojtaś, Jakub Sikora
\affiliations
numery albumów: 283234, 283418 \\
\emails
robert.wojtas.stud@pw.edu.pl, jakub.sikora2.stud@pw.edu.pl
}

\newcommand{\todo}[1]{\textcolor{red}{\textbf{TO DO:} #1}}
\newcommand{\jsc}[1]{\textcolor{blue}{\textbf{JS:} #1}}
\newcommand{\rwc}[1]{\textcolor{green}{\textbf{RW:} #1}}

\begin{document}
\maketitle

\noindent \textbf{Temat projektu}: Grupowanie (G), \\
\textbf{Zestaw danych}: Dane FIFA 2019 Complete Player Dataset (Kaggle) -- grupowanie lub predykcja skuteczności zawodników.


\section{Cel projektu}
\label{sec:cel-projektu}
Celem projektu będzie zbadanie użyteczności algorytmów grupowania w~zadaniu dobierania optymalnej pozycji piłkarzy z~gry FIFA~19 na boisku. Umiejętności poszczególnych zawodników są opisane za pomocą szeregu parametrów, które mają za zadanie obrazować zdolności gry w~ataku, pomocy i~obronie. Zadaniem algorytmu grupowania będzie połączenie w~grupy zawodników o~podobnych umiejętnościach, tak aby na tej podstawie móc wnioskować o~ich optymalnej pozycji.
\section{Zbiór danych}
\label{sec:dane}

W~projekcie wykorzystane zostaną dane ze zbioru \emph{FIFA 19 Complete Player Dataset}~\cite{fifa-dataset} z~serwisu \url{kaggle.com}.


\subsection{Charakterystka zbioru danych}
\label{subsec:charakterystka}
Każdy zawodnik ze zbioru danych posiada ponad 70 użytecznych atrybutów opisujących jego umiejętności gry w~piłkę nożną. Wśród nich znajdują się atrybuty opisujące m.in. kontrolę piłki, przyspieszenie czy celność strzałów z~dystansu. 

Większość atrybutów jest opisana za pomocą wartości całkowitych z~zakresu $[0, 99]$. Niektóre atrybuty zostały opisane liczbą całkowitą z~zakresu $[0,5]$. W~zestawie danych, pojawiły się atrybuty ciągłe opisujące wartość zawodnika, jego tygodniówkę, wzrost, wiek i~wagę oraz atrybuty kategoryczne opisujące preferowaną nogę (lewa lub prawa), typ budowy ciała oraz jego główną pozycję.

Każdemu zawodnikowi przypisany jest atrybut kategoryczny opisujący główną pozycję na boisku oraz 26 wskaźników numerycznych, określających jak dobry jest dany zawodnik na poszczególnych pozycjach.

Dodatkowo, zbiór danych został uzupełniony o~metadane opisujące samego zawodnika, takie jak numer na koszulce, nazwisko, link do zdjęcia, drużyna, narodowość czy międzynarodową reputację. 

\begin{figure}[t]
    \centering    
    \begin{tabular}{cc}
    \subfloat[\emph{Short Passing}]{
        \begin{tikzpicture}[thick, scale=0.4]
            \Huge
            \begin{axis}[ymin=0, area style]
                \addplot+[ybar interval,mark=no] plot coordinates { (0, 0) (5, 3) (10, 53) (15, 242) (20, 485) (25, 632) (30, 465) (35, 397) (40, 503) (45, 766) (50, 1350) (55, 2408) (60, 3553) (65, 3496) (70, 2309) (75, 1084) (80, 336) (85, 65) (90, 12) (95, 0) };
            \end{axis}
        \end{tikzpicture}
    } &
    \subfloat[\emph{Strength}]{
        \begin{tikzpicture}[thick, scale=0.4]
            \Huge
            \begin{axis}[ymin=0, area style]
                \addplot+[ybar interval,mark=no] plot coordinates { (0, 0) (5, 0) (10, 0) (15, 1) (20, 2) (25, 40) (30, 301) (35, 354) (40, 618) (45, 800) (50, 1121) (55, 2015) (60, 2631) (65, 3093) (70, 2810) (75, 2354) (80, 1182) (85, 539) (90, 294) (95, 4) };
            \end{axis}
        \end{tikzpicture}    
    } \\
    \subfloat[\emph{Agility}]{
        \begin{tikzpicture}[thick, scale=0.4]
            \Huge
            \begin{axis}[ymin=0, area style]
                \addplot+[ybar interval,mark=no] plot coordinates { (0, 0) (5, 0) (10, 1) (15, 11) (20, 108) (25, 201) (30, 682) (35, 667) (40, 639) (45, 805) (50, 1156) (55, 1813) (60, 2308) (65, 2770) (70, 2734) (75, 2216) (80, 1116) (85, 607) (90, 320) (95, 5) };
            \end{axis}
        \end{tikzpicture}
    } &
    \subfloat[\emph{GK Positioning}]{
        \begin{tikzpicture}[thick, scale=0.4]
            \Huge
            \begin{axis}[ymin=0, area style, ytick={0, 2000, 5000, 8000}]
                \addplot+[ybar interval, mark=no] plot coordinates { (0, 81) (5, 6337) (10, 7876) (15, 1831) (20, 4) (25, 1) (30, 3) (35, 3) (40, 31) (45, 88) (50, 227) (55, 326) (60, 460) (65, 450) (70, 251) (75, 127) (80, 51) (85, 11) (90, 1) (95, 0) };
            \end{axis}
        \end{tikzpicture}
    } \\

    \end{tabular}
    \caption{Przykładowe rozkłady wartości atrybutów z~podanego zbioru danych}
    \label{fig:rozklady-atrybutow}
\end{figure}

Na rysunku~\ref{fig:rozklady-atrybutow} zostały zaprezentowane przykładowe rozkłady wartości badanych atrybutów. W~głównej mierze, są to rozkłady normalne lub gruboogonowe. Ciekawy rozkład mają atrybuty opisujące umiejętności bramkarskie. Z~racji nietypowej specyfiki tej pozycji, atrybuty te mają rozkład dwumodalny. Wysoki lewy pik rozkładu opisuje zawodników z~pola, których umiejętności bramkarskie są niewielkie ale jest ich zdecydowanie więcej. Prawy niewielki pik opisuje rzeczywistych bramkarzy, którzy mają niewątpliwie większe umiejętności, jednak jest ich wyraźnie mniej. Obecność bramkarzy w~zestawie danych jest również jednym z~powodów dla którego część rozkładów ma gruby ogon po lewej stronie. 

\subsection{Przygotowanie atrybutów}
\label{subsec:atrybuty}
\emph{Implementacja ładowania, wstępnego przetwarzania danych oraz selekcji atrybutów została umieszczona w~pliku \texttt{data-preparation.R}} \\

W~pierwszej kolejności, z~racji niewielkiej liczby brakujących wartości, usunęliśmy wszystkie przykłady, które nie posiadały wszystkich wypełnionych kolumn. Ze zbioru należy usunięte zostały atrybuty opisowe, określające nieinteresujące cechy takie jak narodowość, aktualna drużyna, reputacja na świecie czy imię oraz nazwisko. Dodatkowo, ze zbioru danych zostały usunięte atrybuty opisujące numeryczną wartość opisującą umiejętność gry danego zawodnika na danej pozycji. Informacja tego typu znacząco upraszcza zadanie, sprowadzając zadanie do wyboru najwyższej wartości z~podanego zbioru. 

W kolejnym kroku, atrybuty nominalne, niosące ze sobą pewną informację należało sprowadzić do postaci numerycznej. Atrybut binary \emph{Preferred.Foot} został zamieniony na wartość liczbową ze zbioru $[0,1]$. Atrybut \emph{Work.Rate}, opisujący zaangażowanie danego zawodnika w~grę w~ataku oraz w~obronie, posiadał dziewięć nominalnych wartości, takich jak: \texttt{Low/ Medium}, \texttt{High/ Low}, \texttt{High/ High} itp.. Zdecydowaliśmy się rozbić ten atrybut na dwa atrybuty porządkowe \emph{Work.Rate.Offensive} oraz \emph{Work.Rate.Defensive}, które mogły przyjmować wartości $[$\texttt{Low}, \texttt{Medium}, \texttt{High} $]$. Ostatecznie, wartości te zostały sprowadzone do wartości numerycznych, za pomocą polecenia \texttt{as.numeric}.

Kolejnym rozważanym atrybutem, był \emph{Body.Type}. Pierwotnie, posiadał on dziesięć różnych wartości, jednak po zagłębieniu się w~zbiór danych, okazało się że siedem z~nich to wartości unikalne w~całym zbiorze. Wynikają one ze specyfiki gry, w~której wyróżniające się postacie (np. Lionel Messi czy Cristiano Ronaldo), otrzymały własny model postaci. Zgodnie z~wiedzą ekspercką, zminimalizowliśmy zbiór wartości do trzech wartości nominalnych $[$\texttt{Lean}, \texttt{Normal}, \texttt{Stocky}$]$, które ostatecznie również zamieniliśmy na wartości numeryczne. 

Następnym krokiem we~wstępnym przetwarzaniu była konwersja wzrostu (zapisanego w~stopach) na metry oraz wagi (zapisanej w~funtach) na kilogramy oraz policzenie grupującego wskaźnika \emph{Body-Mass Index}. Ostatecznie, każdy z~przygotowanych atrybutów został przeskalowany w~taki sposób, aby jego średnia wartość wynosiła~$0$, a~odchylenie standardowe było równe~$1$. W~ten sposób, otrzymaliśmy obiekt typu \emph{data frame} zawierający $45$ zmiennych oraz $18159$ przykładów.

Jako atrybut ukryty, potraktowana zostania optymalna pozycja na boisku. W~celu przeprowadzenia analizy poprawności badanego algorytmu grupowania, zakładamy że pozycja na boisku zapisana w~zbiorze danych jest jego rzeczywistą optymalną pozycją. Atrybut ten jest niezwykle ciekawy do analizy, ponieważ można go analizować na trzech poziomach:

\begin{itemize}
    \item na podstawie gry w~polu:
        \begin{itemize}
            \item bramkarz,
            \item zawodnik z~pola,
        \end{itemize}
    \item na podstawie ogólnej pozycji:
        \begin{itemize}
            \item bramkarz,
            \item obrońca,
            \item pomocnik,
            \item napastnik,
        \end{itemize}
    \item na podstawie szczegółowej pozycji:
    \begin{itemize}
        \item bramkarz,
        \item środkowy obrońca, 
        \item lewy/prawy boczny obrońca
        \item lewy/prawy wahadłowy,
        \item pomocnik ofensywny,
        \item lewy/środkowy/prawy pomocnik,
        \item pomocnik defensywny,
        \item napastnik,
        \item lewy/prawy skrzydłowy.
    \end{itemize}
\end{itemize}

Oprócz tego, w~zależności od wyników grupowania, można spróbować zdefiniować bardziej wyrafinowane pozycje m.in.: 
\emph{mezzala}, \emph{mediano}, \emph{trequartista} czy libero.

\subsection{Wstępne ograniczenie liczby atrybutów}
\label{subsec:ograniczanie-atrybtow}
W~celu skrócenia czasu obliczeń oraz wyeliminowania niepewności związanej z~za dużą liczbą zmiennych, zdecydowaliśmy się ograniczyć liczbę atrybutów z~$45$ do~$20$. Liczba ta została określona na podstawie macierzy korelacji atrybutów. Na jej podstawie, wybraliśmy zestawy atrybutów, które połączyliśmy ze sobą poprzez policzenie ich średnich arytmetycznych. Tak skonstruowane atrybuty zostały przedstawione w~tabeli~\ref{tab:nowe-atrybuty}.

\begin{table}[h]
\centering
\begin{tabular}{|c|l|}
\hline
\textbf{Nowy atrybut} & \multicolumn{1}{c|}{\textbf{Atrybuty pierwotne}}                  \\ \hline
GK.Skills             & \multicolumn{1}{p{5cm}|}{\raggedright GKDiving, GKHandling, GKKicking, GKPositioning, GKReflexes}        \\ \hline
Tackling              & \multicolumn{1}{p{5cm}|}{\raggedright Interceptions, StandingTackle, SlidingTackle, Aggression, Marking} \\ \hline
Swiftness             & \multicolumn{1}{p{5cm}|}{\raggedright SprintSpeed, Acceleration}                                         \\ \hline
Short.Ball.Skills     & \multicolumn{1}{p{5cm}|}{\raggedright BallControl, ShortPassing, Dribbling, Skill.Moves}                 \\ \hline
Intelligence          & \multicolumn{1}{p{5cm}|}{\raggedright Positioning, Vision, Composure}                                   \\ \hline
Shooting              & \multicolumn{1}{p{5cm}|}{\raggedright Finishing, Volleys, LongShots}                                     \\ \hline
Headers               & \multicolumn{1}{p{5cm}|}{\raggedright HeadingAccuracy, Jumping}                                          \\ \hline
Free.Kicks            & \multicolumn{1}{p{5cm}|}{\raggedright FKAccuracy, Curve}                                             \\ \hline
\end{tabular}
\caption{Nowe atrybuty wygenerowane na podstawie łączenia wysoko skorelowanych atrybutów pierwotnych~\label{tab:nowe-atrybuty}}
\end{table}

Ostateczny zbiór atrybutów wyglądał następująco:
\begin{itemize}
    \item \emph{Age} -- wiek zawodnika,
    \item \emph{Preferred.Foot} -- lepsza noga,
    \item \emph{Weak.Foot} -- umiejętność gry słabszą nogą,
    \item \emph{Crossing} -- dośrodkowania,
    \item \emph{Long.Passing} -- długie podania,
    \item \emph{Reactions} -- szybkość reakcji,
    \item \emph{Balance} -- balans ciała,
    \item \emph{Penalties} -- pewność wykonywania rzutów karnych,
    \item \emph{Work.Rate.Offensive} -- zaangażowanie w ataku,
    \item \emph{Work.Rate.Defensive} -- zaangażowanie w obronie,
    \item \emph{BMI} -- Body-Mass Index,
    \item \emph{Body.Type} -- budowa ciała,
    \item \emph{GK.Skills} -- umiejętności bramkarskie,
    \item \emph{Tackling} -- przejmowanie piłki,
    \item \emph{Swiftness} -- zwinność w~poruszaniu się,
    \item \emph{Short.Ball.Skills} -- gra na małym obszarze,
    \item \emph{Intelligence} -- inteligencja boiskowa,
    \item \emph{Shooting} -- wykończenie,
    \item \emph{Headers} -- gra głową,
    \item \emph{Free.Kicks} -- pewność wykonywania rzutów wolnych.
\end{itemize}
\section{Badane algorytmy}
\label{sec:algorytmy}
W~ramach podstawowej częśći projektu, zdecydowaliśmy się na zbadanie trzech typów algorytmów grupowania bazujących na niepodobieńtswie przykładów: $k$-medoidów, hierarchicznego oraz rozmytego.

\subsection{Algorytm k-medoidów}
\label{subsec:kmeans}
Algorytm $k$-średnich jest jednym z~najbardziej podstawowych mechanizmów grupowania. Jego jest pogrupowanie danych na $k$ grup w~taki sposób, aby zminimalizować sumę kwadratów odległości euklidesowych punktów od wyznaczonych centroidów (środków) grup. Wyznaczone centroidy są średnią wszystkich punktów należących do jego grupy, co oznacza że centroid nie jest ściśle związany z~żadnym przykładem oraz że algorytm może być wrażliwy na wartości odstające oraz anomalie~\cite{zum}.

Z~tego powodu, w~projekcie rozważony zostanie algorytm $k$-medoidów, który przyporządkowuje centroidom jeden z~grupowanych przykładów. Co więcej, w~odróżnieniu od pierwotnej wersji, algorytm ten gwarantuje zbieżność niezależnie od zastosowanej miary niepodobieństwa.

W~pierwszym kroku algorytmu należy dobrać liczbę poszukiwanych grup oraz przyporządkować w~sposób losowy wszystkie obserwacje do grup. Następnie, w~$i$-tym kroku należy obliczyć położenie centroidu dla każdej grupy i~dla każdego punktu zaktualizować przynależności do grupy. Algorytm kończy się gdy w~danym kroku nie została zmieniona żadna przynależność lub przekroczono maksymalną liczbę kroków.

Domyślnie algorytm $k$-medoidów jest wywoływany za pomocą polecenia \texttt{pam}. Polecenie to znajduje sie~w pakiecie \texttt{cluster}.

Głównym parametrem algorytmu $k$-średnich jest liczba szukanych grup $k$. Liczbę tą podajemy $a priori$, przed uruchomieniem algorytmu. Jednym ze sposobów na odpowiedni dobór parametru $k$ jest przeszukiwanie zupełne w~rozsądnie dobranym przedziale i~znalezienie takiej wartości, dla której badany wskaźnik jakości uzyskuje najlepszą wartość.

Największym problemem algorytmu $k$-medoidów jest dobór początkowego podziału na grupy. Możliwymi sposobami są losowy wybór $k$ przykładów ze zbioru danych lub ręczne podanie zbioru centroidów początkowych.

\subsection{Algorytmy grupowania hierarchicznego}
\label{subsec:hierarchical}
Grupowanie hierarchiczne w eksploracji danych i statystyce jest metodą analizy, która ma na celu zbudowanie hierarchii klastrów. W przeciwieństwie do wielu algorytmów dzielących zbiory danych na klastry w tym wypadku nie jest konieczne wstępne określenie liczby tworzonych klastrów. Algorytmy grupowania  hierarchicznego dzielimy na dwa typy: aglomeracyjne (łączące) i deglomeracyjne (dzielące) \cite{maimonrokach}. Wynikiem użycia metod hierarchicznego grupowania jest zestaw zagnieżdżonych klastrów, które są zwykle prezentowane na~dendrogramie. Dendrogram jest to wielopoziomowa hierarchia, w której klastry z jednego poziomu są połączone i tworzą większe klastry na kolejnych poziomach.

Przy grupowaniu hierarchicznym wykorzystywane są różne sposoby na określanie połączenia między dwoma klastrami. Są to miary odległości pomiędzy dwoma klastrami. Do najbardziej popularnych, możliwych do wyboru w algorytmach wbudowanych w język R należą\cite{rodriguez2015}:

\begin{itemize}
    \item \textbf{pojedyncze połączenie} - minimalna odległość między obserwacją w jednym klastrze a obserwacją w innym klastrze,
    \item \textbf{kompletne połączenie} - maksymalna odległość między obserwacją w jednym klastrze a obserwacją w innym klastrze,
    \item \textbf{średnie połączenie} - średnia odległość między obserwacją w jednym klastrze a obserwacją w innym klastrze,
    \item \textbf{połączenie centroidalne} - odległość pomiędzy centroidami klastrów,
    \item \textbf{połączenie medianowe} - mediana odległości między obserwacją w jednym klastrze a obserwacją w innym klastrze
\end{itemize}


Należy pamiętać o tym, że w zależności od zbioru danych, niektóre metody mogą działać lepiej od innych. W projekcie podczas eksperymentów badane były połączenia: kompletne i średnie. 

\subsubsection{Algorytmy aglomeracyjne}
Metody aglomeracyjne są najbardziej popularną metodą grupowania hierarchicznego. Zaraz po rozpoczęciu algorytmu każdy obiekt ze zbioru danych jest traktowany jako pojedynczy klaster. Kolejno klastry są  łączone, do momentu gdy wszystkie klastry zostaną scalone w jeden duży klaster zawierający wszystkie obiekty zbioru danych. W języku R dokonujemy grupowania algorytmem aglomeracyjnym przy użyciu polecenia \texttt{agnes}.

\subsubsection{Algorytmy deglomeracyjne}
Metody deglomeracyjne są przeciwieństwem metod aglomeracyjnych. Na początku działania algorytmu wszystkie obserwacje znajdują się w jednym klastrze. W kolejnych iteracjach klastry są dzielone na mniejsze \cite{kassambara}. Proces dzielenia klastrów powtarzany jest do momentu, gdy każda obserwacja znajduje się we właściwym klastrze tj. do momentu gdy liczba klastrów będzie równa liczbie obserwacji. W języku R dokonujemy grupowania deglomeracyjnego przy użyciu polecenia \texttt{diana}.

\subsection{Algorytmy grupowania rozmytego}
\label{subsec:fuzzy}
Algorytmy grupowania rozmytego bazują na założeniu że pojedyncza obserwacja może równocześnie należeć do wielu grup. To jak bardzo przykład należy do danej grupy $v$, opisuje \emph{funkcja przynależności} $\mu_{v}(x)$, która może przyjmować wartości z~przedziału $[0,1]$. Dla pojedynczej obserwacji $x$ spełnione są następujące własności:

\begin{equation}
    \sum_{v=1}^{k} \mu_{v}(x)    
\end{equation}

\begin{equation}
    \forall{v} \quad \mu_{v}(x) \geq 0     
\end{equation}

Podejście rozmyte może potencjalnie dać więcej korzyści, biorąc pod uwagę temat projektu. W~nowoczesnym futbolu, co raz częściej możemy spotkać się z~sytuacją w~której zawodnicy muszą umieć grać na kilku (jak nie wszystkich) pozycjach równocześnie.

W~projekcie porównamy działanie dwóch algorytmów rozmytego grupowania: fuzzy C-means oraz FANNY.

Podstawowym algorytmem grupowania rozmytego jest rozszerzenie algorytmu $k$-średnich na zbiory rozmyte, znany w~literaturze jako rozmyte $k$-średnich lub fuzzy C-means~\cite{cmeans}.

Algorytm ten nie różni się wiele od wersji oryginalnej. Jedyną różnicą jest zmiana sposobu określania przynależności do grupy, zamiast szukania najbliższej grupy, obliczamy przynależność do wszystkich grup równocześnie. W~pierwszym kroku algorytmu należy dobrać liczbę poszukiwanych grup oraz przyporządkować w~sposób losowy początkowe wartości funkcji przynależności do wszystkich grup. Następnie, w~$i$-tym kroku należy obliczyć położenie centroidu dla każdej grupy, a~dalej dla każdego punktu zaktualizować funkcje przynależności do wszystkich grup. Algorytm kończy działanie w~momencie gdy wartości funkcji przynależności zmieniły się o~wartość mniejszą niż $\epsilon$ lub osiągnięta została maksymalna liczba kroków.

Algorytm rozmytych $k$-średnich został zaimplementowany w~pakiecie \texttt{cluster} jako funkcja \texttt{fcm}. Parametrami wymagającymi strojenia jest liczba grup, zadana miara niepodobieństwa, maksymalna liczba iteracji oraz stopień rozmycia~\cite{cmeans}.

Drugim badanym przez nas algorytmem będzie algorymt FANNY opracowany przez Kaufmana i~Rousseeuwa~\cite{kaufman2009finding} a~rozszerzony przez Martina Maechlera. Algorytm ten minimalizuje następującą funkcję celu:

\begin{equation}
    \sum_{v=1}^{k} \frac{\sum_{i=1}^{n}\sum_{j=1}^{n} \mu^{r}_{v}(x_{i}) \mu^{r}_{v}(x_{j}) d(x_{i}, x_{j})}{2 \sum_{j=1}^{n} \mu^{r}_{v}(x_{j})} 
\end{equation}

gdzie~$n$ oznacza liczbę obserwacji w~zbiorze danych, $k$~jest liczbą grup, $r$~jest \emph{wykładnikiem przynależności}, natomiast~$d(x_{i}, x_{j})$ jest wybraną funkcją niepodobieństwa przykładów~$x_{i}$~i~$x_{j}$.

Algorytm ten został zaimplementowany w~pakiecie \texttt{cluster} jako funkcja \texttt{fanny}. Parametrami tej funkcji są: liczba grup, wykładnik przynależności, funkcja odległości oraz maksymalna liczba iteracji.
\section{Miary}
\label{sec:miary}

\subsection{Miary niepodobieństwa}
\label{subsec:podobienstwo}
Wszystkie opisane wcześniej algorytmy automatycznego grupowania bazują na pewnej funkcji $d(x_{i}, x_{j})$ zwanej miarą niepodobieństwa lub odległości. Opisuje ona jak \emph{różne} są od siebie dwa przykłady $x_{i}, x_{j})$. 

Najbardziej podstawową miarą niepodobieństwa jest odległość euklidesowa. Jest to pierwiastek z~sumy kwadratów różnic pomiędzy poszczególnymi atrybutami lub też długość odcinka łączącego dwa punkty. Odległość ta dana jest wzorem:

\begin{equation}
    d_{e}(x, y) = \left( \sum_{k=1}^{n} (x_k - y_k)^{2} \right)^{\frac{1}{2}}
\end{equation}

Inną interesującą miarą podobieństwa jest odległość Manhattan. Bierze ona swoją nazwę od regularnej siatki ulic w~dzielnicy Manhattan w~Nowym Yorku. Odległość dwóch punktów w tej metryce to suma wartości bezwzględnych różnic ich współrzędnych.

\begin{equation}
\label{eqn:manhattan}
d_{manh}(x, y) = \sum_{k=1}^{n}|x_k - y_k|
\end{equation}

Ogólną wersją przedstawionych miar jest odległość Minkowskiego~\cite{irani2016clustering}. Definiowana jest poprzez następujący wzór:

\begin{equation}
    d_{mink}(x, y) = \left( \sum_{k=1}^{n} |x_k - y_k|^{p} \right)^{\frac{1}{p}}
\end{equation}

Poprzez manipulowanie parametrem $p$ możemy wpływać na to jak wartość bezwzględna  pomiędzy poszczególnymi składowymi skalarnymi, wpływa na całkowitą wartość funkcji miary. Dla $p=1$ otrzymujemy odległość Manhattan a~dla $p=2$ otrzymujemy odległość euklidesową.

Z~racji weryfikowanej hipotezy, przydatna może okazać się miara niepodobieństwa, która bierze pod uwagę układ wartości wysokich i~niskich, a~nie same wielkości różnić pomiędzy składowymi. W~związku z~tym, wykorzystana zostanie odległość bazująca na korelacji dwóch wektorów. Korelacja osiąga wartość zerową dla maksymalnie niepodobnych do siebie przykładów, dlatego też miara niepodobieństwa wymaga przekształcenia do postaci:

\begin{equation}
    d_{corr}(x, y) = 1 - corr(x,y)
\end{equation}

Wszystkie rozważane miary podobieństwa zostały zaimplementowane w~pakiecie \texttt{proxy} i~można je wywołać za pomocą polecenia \texttt{dist}.

\subsection{Miary jakości grupowania}
\label{subsec:jakosc}
Aby móc określić jak \emph{dobre} jest uzyskane przyporządkowanie do grup, należy policzyć miary jakości grupowania. Można wyróżnić dwa typy miar: zewnętrzne oraz wewnętrzne,

\subsubsection{Ocena zewnętrzna}
Miary zewnętrzne wykorzystują informację o~atrybucie ukrytym, jednak nie w~takim stopniu jak typowe miary klasyfikacji jak na przykład macierz pomyłek (i~wszystkie powiązanie z~nią wskaźniki). W~ogólności, oceniają one zgodność uzyskanego przyporządkowania z~zewnętrznym etykietowaniem. W~ramach projektu, zostaną zbadane dwa wskaźniki jakości:
skorygowany indeks Randa oraz odległość współdzielonej informacji (ang. \emph{shared information distance}, \emph{variation of information}).

Skorygowany indeks Randa jest jednym ze sposobów zmierzenia podobieństwa pomiędzy dwoma etykietowaniami. Aby otrzymać ten wskaźnik, należy w~pierwszej kolejności zbudować następującą tabele przyporządkowań: 

\begin{equation*}
\centering
\begin{array}{c|cccc|c}
{{} \atop X}\!\diagdown\!^Y &
Y_1 & Y_2& \cdots& Y_s& \text{sums} \\
\hline
X_1 & n_{11} & n_{12} & \cdots & n_{1s} & a_1 \\
X_2 & n_{21} & n_{22} & \cdots & n_{2s} & a_2 \\
\vdots & \vdots & \vdots & \ddots & \vdots & \vdots \\
X_r & n_{r1} & n_{r2} & \cdots & n_{rs} & a_r \\
\hline
\text{sums} & b_1 & b_2 & \cdots & b_s &
\end{array}
\label{eqn:ari-table}
\end{equation*}
\medskip
Na podstawie tabeli, można obliczyć wskaźnik z~następującego wzoru:
\begin{equation*}
    ARI = \frac{ \left. \sum_{ij} \binom{n_{ij}}{2} - \left[\sum_i \binom{a_i}{2} \sum_j \binom{b_j}{2}\right] \right/ \binom{n}{2} }{ \left. \frac{1}{2} \left[\sum_i \binom{a_i}{2} + \sum_j \binom{b_j}{2}\right] - \left[\sum_i \binom{a_i}{2} \sum_j \binom{b_j}{2}\right] \right/ \binom{n}{2} }
    \label{eqn:ari}
\end{equation*}
\bigskip

Odległość współdzielonej informacji opisuje odległość pomiędzy dwoma etykietowaniami, wykorzystując do tego informację wzajemną. Metryka ta zachowuje nierówność trójkąta i~dana jest wzorem:

\begin{equation*}
    VI(X, Y) = - \sum_{i,j} r_{i,j} [log(r_{i,j} / p_{i,j}) + log( r_{i,j} / q_{i,j})]
\end{equation*}

Wskaźniki te mogą dać informację jak poprawne jest nasze grupowanie, jednak do ich obliczenia potrzebna będzie informacja o~wartości atrybutu ukrytego. Wskaźniki muszą zostać obliczone trzykrotnie: raz dla analizy na podstawie gry w~polu, raz dla analizy na podstawie ogólnej pozycji i~ostatecznie trzeci raz dla analizy na podstawie szczegółowej pozycji.

Do obliczenia obu wskaźników zostanie wykorzystane polecenie \texttt{cluster.stats} z~pakietu~\texttt{fpc}.

\subsubsection{Ocena wewnętrzna}
Drugim, znacznie ciekawszym, podejściem do problemu jest pominięcie wartości atrybutu ukrytego i~skupienie się na jakości wewnętrznej samego grupowania, poprzez zbadanie miar oceny wewnętrznej. Wartość danej miary jakości jest określana na podstawie samego podziału, bez dodatkowej informacji o~wartości atrybutu ukrytego. Miary tego typu skupiają się na głównie na separowalności samych grup oraz na tym jak są dobrze wewnętrznie spójne i~określone. Metryki tego typu to suma kwadratów odległości wewnątrzgrupowych, wskaźnik sylwetki, wskaźnik Calinskiego-Harabasza oraz indeks Dunna.

Suma kwadratów odległości wewnątrzgrupowych jest najprostszym wskaźnikiem oceny wewnętrznej etykietowania. Aby móc obliczyć ten wskaźnik dla miar niepodobieństwa innych niż odległość euklidesowa, wykorzystana zostanie wersja ogólna wzoru:

\begin{equation}
    D = \frac{1}{2} \sum^{K}_{k=1} \sum^{n_k}_{i=1} d(x_i, \bar{x_{k}})
\end{equation}

Innym, niezwykle popularnym wskaźnikiem oceny wewnętrznej jest wskaźnik sylwetki. Mierzy on spójność przykładów w~obrębie danej grupy oraz jak dobrze jest separowalny względem innych grup. Wysoka wartość wskaźnika sylwetki informuje nas o~dobry grupowaniu. Wskaźnik ten oblicza się dla każdego badanego punktu, wykorzystując wzory:

\begin{equation}
    a(i) = \frac{1}{|C_{i}| - 1} \sum_{j \in C_{i}, i \neq j} d(i,j)
\end{equation}

\begin{equation}
    b(i) = \min_{k \neq i} \frac{1}{|C_k|} \sum_{j \in C_{k}} d(i,j)
\end{equation}

\begin{equation}
    s(i) = \frac{b(i) - a(i)}{max \{ a(i), b(i) \} }
\end{equation}

Aby ocenić jakość całego etykietowania a~nie przyporządkowania jednego punkty, policzony zostanie średni wskaźnik sylwetki, jako średnia arytmetyczna wskaźników dla wszystkich punktów.

Wskaźnik Calinskiego-Harabasza opisuje jak dobrze dla danego etykietowania, określone i~spójne są utworzone grupy. Im wyższa wartość wskaźnika tym bardziej spójny podział. Wskaźnik ten można obliczyć ze wzoru:

\begin{equation}
    s = \frac{tr(B_{k})}{tr(W_k)} \times \frac{n_{E} - k}{k - 1}
\end{equation}

gdzie $B_{k}$ jest macierzą rozproszenia pomiędzy grupami, $W_{k}$ jest macierzą rozproszenia wewnątrzngrupowego a $n_{E}$ to liczba przykładów.

Ostatnim rozważanym wskaźnikiem jest wskaźnik Dunna, dany wzorem:
\begin{equation}
    DI_{m} = \min_{1 \leq i \leq k} \left[ \min_{1 \leq j \leq k} \left(  \frac{\delta(X_i, X_j)}{\max_{1 \leq l \leq k} \Delta(X_l)}   \right)     \right]
\end{equation}
gdzie $\delta(X_i, X_j)$ równa jest odległości międzygrupowej pomiędzy $X_i$ oraz $X_j$ a $\Delta(X_l)$ równa jest sumie odległości wewnątrzgrupowych w~grupie $X_l$. Wskaźnik ten zwraca wysokie wartości dla etykietowań spójnych, kompaktowych i~dobrze separowalnych.

Na podstawie miar tego typu można spróbować dobrać najlepszą ilość grup na jaką można podzielić zbiór danych. Im mniej grup, tym więcej informacji tracimy. Za duża liczba grup spowoduje nadmierne dopasowanie. 

Do obliczenia wszystkich wymienionych wyżej wskaźników również zostanie wykorzystane polecenie \texttt{cluster.stats} z~pakietu~\texttt{fpc}.

\subsubsection{Miary rozmyte}
Przedstawione powyżej miary jakości, odnoszą się wyłącznie do klasycznego grupowania, w~którym danemu przykładowi, przyporządkowywana jest dana grupa. W~przypadku algorytmów grupowania rozmytego, należy zastosować specjalnie przystosowane miary, biorące pod uwagę niebinarną funkcję przynależności.

W~ramach projektu, zbadany zostanie wskaźnik średniej rozmytej szerokości sylwetki~\cite{fsil}. Jest on obliczany zgodnie ze~wzorem: 

\begin{equation}
        FS = \frac{\sum_{j=1}^N ( \mu_{pj} - \mu_{qj} )^{\alpha} s_{j}} { \sum_{j=1}^N ( \mu_{pj} - \mu_{qj} }
\end{equation}

Wskaźnik ten korzysta z~parametru $\alpha$, który określa stopnie fuzyfikacji oraz z~$\mu_{pj}$,$\mu_{qj}$, które są kolejno pierwszym i~drugim największym elementem w~$j$-tej kolumnie macierzy rozmytego podziału. W~ten sposób, większy nacisk jest położony na obiekty znajdujące się blisko środka podziału, a~ignorowane są punkty znajdujące się w~obszarze nakładania się dwóch grup~\cite{fsil}. Wskaźnik ten został zaimplementowany jako funkcja \texttt{F.SIL} pakietu \texttt{fclust}~\cite{fclust}.
\section{Wyniki badań}
\label{sec:wyniki}
Do przeprowadzenia eksperymentów posłużyły napisane w języku R skrypty zawierające instrukcje i wykonujące polecenia języka~R, niezbędne do wykonania grupowania na zbiorze danych. Dla każdego algorytmu został utworzony osobny skrypt, którego wynikiem wykonania, były dane opisujące przeprowadzone przyporządkowanie do grup. 

Omawiane wcześniej miary jakości grupowania obliczane były dla różnej ilości grup, zwiększanej iteracyjne w miarę postępowania eksperymentów. Grupowanie zostało przeprowadzone z~zadaną liczbą grup z~zakresu od dwóch do czterdziestu. Ilość potrzebnych uruchomień algorytmów w połączeniu z dużą ilością danych w zbiorze i kilkoma miarami odległości skutkowała bardzo długim czasem wykonania skryptów. W celu przyspieszenia wykonania eksperymentów, standardowe podejście z~wykonaniem algorytmu na całym zbiorze danych zastąpiono nieco innym postępowaniem.

Ze zbioru piłkarzy wydzielono 10 procent danych. Tę część danych wykorzystano do wyznaczenia wstępnego grupowania z~wykorzystaniem danego algorytmu grupowania. Następnie wyznaczano środki otrzymanych grup. Finalne przyporządkowanie do grup otrzymywano poprzez dopasowanie elementów do wyznaczonych centroidów, wykorzystując badaną aktualnie miarę odległości.

%%%%%%%%%%%%%%%%%%%%%%%%%%%%%%%%%%%%%%%%%%%%%%%%%%%%%%%%%%%%%%%%%%%%%%%%%%%%%%%%%%%%%%%%%%%%%%%%%%%%%%%%%%%%%%%%%%%%%%%%%%%%%%%%%%%%%%%%%%%%%%%%

\subsection{Algorytm k-medoidów}
\label{subsec:wyniki-kmedoids}
\emph{Kod realizujący poniżej przedstawione eksperymenty został umieszczony w~pliku \texttt{k-medoids.R}}.

W~celu zbadania efektów grupowania danych przy użyciu algorytmu k-medoidów oraz doboru optymalnej liczby grup, porównaliśmy ze sobą wartości trzech wskaźników jakości dla wszystkich czterech badanych miar niepodobieństwa, a~także kształty krzywych opisujących sumę kwadratów odległości wewnątrzgrupowych. 

\begin{figure}[t]
    \centering
    \captionsetup[subfloat]{farskip=2pt,captionskip=1pt}
    
    \subfloat[Miara euklidesowa]{\includegraphics[width=0.5\columnwidth]{./figures/full/kmed/ss.euclidean.pdf}}
    \subfloat[Miara manhattan]{\includegraphics[width=0.5\columnwidth]{./figures/full/kmed/ss.manhattan.pdf}}\hfill
    
    \subfloat[Miara Minkowskiego]{\includegraphics[width=0.5\columnwidth]{./figures/full/kmed/ss.minkowski.pdf}}
    \subfloat[Miara korelacyjna]{\includegraphics[width=0.5\columnwidth]{./figures/full/kmed/ss.correlation.pdf}}\hfill
    \caption{Porównanie krzywych sumy kwadratów odległości przy grupowaniu algorytmem k-medoidów dla różnych miar niepodobieństwa}
    \label{fig:kmed-full-ss}
\end{figure}

Na podstawie krzywych, przedstawionych na rysunku~\ref{fig:kmed-full-ss}, ciężko ocenić optymalną liczbę grup, ponieważ na pierwszy rzut oka ciężko dostrzec punkt łokciowy. Po dokładnym przyjrzeniu się można spróbować ocenić że optymalne $k$ jest bardzo niskie i~zamyka się w~przedziale $[2,3]$. 

Na rysunku~\ref{fig:kmed-full-sil} porównane zostały przebiegi wartości średniej szerokości sylwetki, w~zależności od zadanej liczby grup. Krzywe: niebieska, czerwona oraz zielona z~racji bliźniaczej natury, przedstawiają podobny, mało zadowalający przebieg. Charakterystyka krzywej odpowiadającej mierze korelacyjnej posiada zupełnie inny charakter. Zmiana parametru $k$ zdaje się wpływać na nią w~niewielkim stopniu.

\begin{figure}[h]
    \centering
    \includegraphics[width=1.1\columnwidth]{./figures/full/kmed/sil.widths.pdf}
    \caption{Porównanie średniej szerokości sylwetki dla grupowania metodą k-medoidów dla różnych miar niepodobieństwa}
    \label{fig:kmed-full-sil}
\end{figure}

Rysunek~\ref{fig:kmed-full-ch} przedstawia zależność wskaźnika Callinskiego-Harabasza w~zależności od liczby grup. Wraz z~rosnącą liczbą grup, wartość wskaźnika maleje. Z~racji braku górnej wartości wskaźnika, możemy jedynie stwierdzić która z~miar odległości lepiej nadaje się do tego zadania. Z~rysunku wynika, że najlepszą jakość grupowania osiągnięto dla odległości Manhattan, natomiast najgorszą dla miary korelacyjnej.
 
\begin{figure}[h]
    \centering
    \includegraphics[width=1.1\columnwidth]{./figures/full/kmed/ch.pdf}
    \caption{Porównanie wskaźnika Callinskiego-Harabasza dla grupowania metodą k-medoidów dla różnych miar niepodobieństwa}
    \label{fig:kmed-full-ch}
\end{figure}

Ostatnim badanym wskaźnikiem jakości wewnętrznej jest wskaźnik Dunna, którego zależność od liczby grup została przedstawiona na rysunku~\ref{fig:kmed-full-dunn}. Dla wszystkich czterech miar niepodobieństwa, wartość wskaźnika Dunna utrzymuje się na stałym, bardzo niskim poziomie, co wskazuje na niską jakość grupowania.

\begin{figure}[h]
    \centering
    \includegraphics[width=1.1\columnwidth]{./figures/full/kmed/dunn.pdf}
    \caption{Porównanie wskaźnika Dunna dla grupowania metodą k-medoidów dla różnych miar niepodobieństwa}
    \label{fig:kmed-full-dunn}
\end{figure}

Uzyskane wyniki wskazują jednoznacznie, że algorytm k-medoidów nie poradził sobie z~grupowaniem podanych danych. Najlepszy uzyskany podział udało się uzyskać dla parametru $k=2$. Podział taki dobrze opisuje pierwszy sposób opisu danych, czyli \emph{na podstawie gry w~polu}. Aby sprawdzić jak dobrze uzyskane etykietowania opisują ten podział, zostanie zbadany skorygowany wskaźnik Randa, którego wartości wskaźnika zostały przedstawione na rysunku~\ref{fig:kmed-full-rand-outfield}. 

Dla takiego podziału, skorygowany wskaźnik Randa, uzyskuje najwyższą wartość dla klasycznych miar Minkowskiego i~uzyskuje maksymalną możliwą wartość. Grupowanie, mimo niskiej jakości wewnętrznej, poprawnie sklasyfikowało wszystkie przykłady ze zbioru. Miara korelacyjna, podobnie jak przy miarach wewnętrznych, wypada najsłabiej, a~wartość indeksu oscyluje w~okolicach zera.

\begin{figure}[h]
    \centering
    \includegraphics[width=1.1\columnwidth]{./figures/full/kmed/rand-outfield.pdf}
    \caption{Porównanie skorygowanego wskaźnika Randa dla grupowania metodą k-medoidów dla różnych miar niepodobieństwa przy podziale na podstawie gry w~polu}
    \label{fig:kmed-full-rand-outfield}
\end{figure}

Miara odległości współdzielonej informacji dla miar klasycznych osiąga wartość zerową, co oznacza stuprocentowe dopasowanie etykietowań. Zgodnie z~oczekiwaniami, wartość tej miary rośnie, wraz z~rosnącą liczbą grup. Pełna zależność dla wszystkich grup została przedstawiona na rysunku~\ref{fig:kmed-full-vi-outfield}.

\begin{figure}[h]
    \centering
    \includegraphics[width=1.1\columnwidth]{./figures/full/kmed/vi-outfield.pdf}
    \caption{Porównanie skorygowanego wskaźnika Randa dla grupowania metodą k-medoidów dla różnych miar niepodobieństwa przy podziale na podstawie gry w~polu}
    \label{fig:kmed-full-vi-outfield}
\end{figure}

Dla innych podziałów, jakość grupowania nie jest już tak zadowalająca. Na rysunku~\ref{fig:kmed-full-rand-specific} przedstawiono zależność skorygowanego indeksu Randa od liczby grup przy podziale na podstawie szczegółowej pozycji. Miary klasyczne osiągają swoje niewielkie maksimum pomiędzy $10$ a $14$ grupami, wstrzeliwując się w~oczekiwaną liczbę etykietowań. Zastanawiająca jest charakterystyka indeksu przy zastosowaniu miary korelacyjnej. Najwyższą wartość osiąga on dla $32$ grup, a~następnie drastycznie spada, co odpowiada spadkowi na wykresie~\ref{fig:kmed-full-sil}.

\begin{figure}[h]
    \centering
    \includegraphics[width=1.1\columnwidth]{./figures/full/kmed/rand-specific.pdf}
    \caption{Porównanie odległości współdzielonej informacji dla grupowania metodą k-medoidów dla różnych miar niepodobieństwa przy podziale na podstawie szczegółowej pozycji}
    \label{fig:kmed-full-rand-specific}
\end{figure}

%%%%%%%%%%%%%%%%%%%%%%%%%%%%%%%%%%%%%%%%%%%%%%%%%%%%%%%%%%%%%%%%%%%%%%%%%%%%%%%%%%%%%%%%%%%%%%%%%%%%%%%%%%%%%%%%%%%%%%%%%%%%%%%%%%%%%%%%%%%%%%%%

\subsection{Algorytm aglomeracyjny}
\label{subsec:wyniki-agnes}
\emph{Kod realizujący poniżej przedstawione eksperymenty został umieszczony w~pliku \texttt{agnes-clustering.R}}.

Algorytm aglomeracyjny pozwala na utworzenie hierarchii połączeń pomiędzy grupowanymi przykładami, dzięki czemu możliwym jest uzyskanie dowolnej liczby grup: od jednej grupy zawierającej wszystkie przykłady ze zbioru, po $n$ grup, z~których każda ma tylko jeden przykład. W~ramach eksperymentów, przeprowadziliśmy grupowanie za pomocą polecenia \texttt{agnes} a~następnie podzieliliśmy wynik ze względu na zadaną liczbę grup poleceniem \texttt{cutree}. W~celu zbadania jakości etykietowania oraz doboru optymalnej liczby grup, ponownie porównaliśmy ze sobą wartości trzech wskaźników jakości dla wszystkich czterech badanych miar niepodobieństwa, a~także kształty krzywych opisujących sumę kwadratów odległości wewnątrzgrupowych. Dodatkowo, zbadaliśmy różnicę pomiędzy metodą łączenia przykładów. W~ramach projektu, skorzystaliśmy z~połączenia kompletnego oraz średniego.

\subsubsection{Połączenie kompletne}

\begin{figure}[h]
    \centering
    \captionsetup[subfloat]{farskip=2pt,captionskip=1pt}
    
    \subfloat[Miara euklidesowa]{\includegraphics[width=0.5\columnwidth]{./figures/full/agnes-complete/ss.euclidean.pdf}}
    \subfloat[Miara Manhattan \label{fig:agnes-complete-full-ss-manhattan}]{\includegraphics[width=0.5\columnwidth]{./figures/full/agnes-complete/ss.manhattan.pdf}}\hfill
    
    \subfloat[Miara Minkowskiego]{\includegraphics[width=0.5\columnwidth]{./figures/full/agnes-complete/ss.minkowski.pdf}}
    \subfloat[Miara korelacyjna \label{fig:agnes-complete-full-ss-correlation}]{\includegraphics[width=0.5\columnwidth]{./figures/full/agnes-complete/ss.correlation.pdf}}\hfill
    \caption{Porównanie krzywych sumy kwadratów odległości przy grupowaniu algorytmem aglomeracyjnym z~połączeniem kompletnym dla różnych miar niepodobieństwa}
    \label{fig:agnes-complete-full-ss}
\end{figure}

Na podstawie krzywych, przedstawionych na rysunku~\ref{fig:agnes-complete-full-ss}, można spróbować dokonać estymacji optymalnej liczby grup. Korzystając z~rysunków~\ref{fig:agnes-complete-full-ss-manhattan} oraz \ref{fig:agnes-complete-full-ss-correlation} wnioskujemy, iż optymalna liczba grup wynosi od $2$ do $5$.

\begin{figure}[H]
    \centering
    \includegraphics[width=1.1\columnwidth]{./figures/full/agnes-complete/sil.widths.pdf}
    \caption{Porównanie średniej szerokości sylwetki dla grupowania metodą aglomeracyjną z~połączeniem kompletnym dla różnych miar niepodobieństwa}
    \label{fig:agnes-complete-full-sil}
\end{figure}

Na rysunku~\ref{fig:agnes-complete-full-sil} porównane zostały przebiegi wartości średniej szerokości sylwetki w~zależności od zadanej liczby grup. Krzywe: niebieska, czerwona oraz zielona przedstawiają podobny, mało zadowalający przebieg, przy czym miara Manhattan wydaje się być najbardziej odporna na zwiększanie liczby grup. Charakterystyka krzywej odpowiadającej mierze korelacyjnej ponownie posiada zupełnie inny charakter od reszty. Zmiana parametru $k$ zdaje się wpływać na nią w~niewielkim stopniu.

\begin{figure}[h]
    \centering
    \includegraphics[width=1.1\columnwidth]{./figures/full/agnes-complete/ch.pdf}
    \caption{Porównanie wskaźnika Callinskiego-Harabasza dla grupowania metodą aglomeracyjną z~połączeniem kompletnym dla różnych miar niepodobieństwa}
    \label{fig:agnes-complete-full-ch}
\end{figure}

Rysunek~\ref{fig:agnes-complete-full-ch} przedstawia zależność wskaźnika Callinskiego-Harabasza w~zależności od liczby grup. Wraz z~rosnącą liczbą grup, wartość wskaźnika maleje.  Z~rysunku wynika również, że najlepszą jakość grupowania osiągnięto dla odległości Manhattan, natomiast najgorszą dla miary korelacyjnej.
 
\begin{figure}[H]
    \centering
    \includegraphics[width=1.1\columnwidth]{./figures/full/agnes-complete/dunn.pdf}
    \caption{Porównanie wskaźnika Dunna dla grupowania metodą aglomeracyjną z~połączeniem kompletnym dla różnych miar niepodobieństwa}
    \label{fig:agnes-complete-full-dunn}
\end{figure}

Ostatnim badanym wskaźnikiem jakości wewnętrznej jest wskaźnik Dunna, którego zależność od liczby grup została przedstawiona na rysunku~\ref{fig:agnes-complete-full-dunn}. Dla wszystkich czterech miar niepodobieństwa, wartość wskaźnika Dunna utrzymuje się na stałym, bardzo niskim poziomie, co wskazuje na niską jakość grupowania. Podobnie jak przy algorytmie k-medoidów, miara korelacyjna również osiąga najgorsze rezultaty.

Uzyskane wyniki wskazują jednoznacznie, że algorytm aglomeracyjny z~połączeniem kompletnym nie poradził sobie z~grupowaniem podanych danych. Najlepszy uzyskany podział udało się uzyskać dla parametru $k=2$. Podział taki dobrze opisuje pierwszy sposób opisu danych, czyli \emph{na podstawie gry w~polu}. Aby sprawdzić jak dobrze uzyskane etykietowania opisują ten podział, zostanie zbadany skorygowany wskaźnik Randa, którego wartości wskaźnika zostały przedstawione na rysunku~\ref{fig:agnes-complete-full-rand-outfield}. 

Dla takiego podziału, skorygowany wskaźnik Randa, uzyskuje najwyższą wartość dla miary Manhattan, która również przejawiała dobre wyniki przy ocenie wewnętrznej. Dla tej miary, grupowanie, mimo niskiej jakości wewnętrznej, poprawnie sklasyfikowało wszystkie przykłady ze zbioru. Miara korelacyjna, podobnie jak przy miarach wewnętrznych, wypada najsłabiej, a~wartość indeksu oscyluje w~okolicach zera. Zastanawiający jest bardzo niski wynik etykietowania bazującego na odległości Minkowskiego.

\begin{figure}[h]
    \centering
    \includegraphics[width=1.1\columnwidth]{./figures/full/agnes-complete/rand-outfield.pdf}
    \caption{Porównanie skorygowanego wskaźnika Randa dla grupowania metodą aglomeracyjną z~połączeniem kompletnym dla różnych miar niepodobieństwa przy podziale na podstawie gry w~polu}
    \label{fig:agnes-complete-full-rand-outfield}
\end{figure}

Miara odległości współdzielonej informacji dla miary Manhattan osiąga wartość zerową, co oznacza stuprocentowe dopasowanie etykietowań. Zgodnie z~oczekiwaniami, wartość tej miary rośnie, wraz z~rosnącą liczbą grup. 

\begin{figure}[H]
    \centering
    \includegraphics[width=1.1\columnwidth]{./figures/full/agnes-complete/vi-outfield.pdf}
    \caption{Porównanie odległości współdzielonej informacji dla grupowania metodą aglomeracyjną z~połączeniem kompletnym dla różnych miar niepodobieństwa przy podziale na podstawie gry w~polu}
    \label{fig:agnes-complete-full-vi-outfield}
\end{figure}

Dla innych podziałów, jakość grupowania nie jest już tak zadowalająca. Na rysunku~\ref{fig:agnes-complete-full-rand-general} przedstawiono zależność skorygowanego indeksu Randa od liczby grup przy podziale na podstawie ogólnej pozycji. Miary klasyczne osiągają swoje niewielkie maksimum dla 4 grup, dobrze opisując oczekiwaną liczbę etykietowań. 

\begin{figure}[h]
    \centering
    \includegraphics[width=1.1\columnwidth]{./figures/full/agnes-complete/rand-general.pdf}
    \caption{Porównanie skorygowanego wskaźnika Randa dla grupowania metodą aglomeracyjną z~połączeniem kompletnym dla różnych miar niepodobieństwa przy podziale na podstawie ogólnej pozycji}
    \label{fig:agnes-complete-full-rand-general}
\end{figure}

%%%%%%%%%%%%%%%%%%%%%%%%%%%%%%%%%%%%%%%%%%%%%%%%%%%%%%%%%%%%%%%%%%%%%%%%%%%%%%%%%%%%%%%%%%%%%%%%%%%%%%%%%%%%%%%%%%%%%%%%%%%%%%%%%%%%%%%%%%%%%%%%

\subsubsection{Połączenie średnie}
\begin{figure}[h]
    \centering
    \captionsetup[subfloat]{farskip=2pt,captionskip=1pt}
    
    \subfloat[Miara euklidesowa]{\includegraphics[width=0.5\columnwidth]{./figures/full/agnes-average/ss.euclidean.pdf}}
    \subfloat[Miara manhattan \label{fig:agnes-average-full-ss-manhattan}]{\includegraphics[width=0.5\columnwidth]{./figures/full/agnes-average/ss.manhattan.pdf}}\hfill
    
    \subfloat[Miara Minkowskiego]{\includegraphics[width=0.5\columnwidth]{./figures/full/agnes-average/ss.minkowski.pdf}}
    \subfloat[Miara korelacyjna \label{fig:agnes-average-full-ss-correlation}]{\includegraphics[width=0.5\columnwidth]{./figures/full/agnes-average/ss.correlation.pdf}}\hfill
    \caption{Porównanie krzywych sumy kwadratów odległości przy grupowaniu algorytmem aglomeracyjnym z~połączeniem średnim dla różnych miar niepodobieństwa}
    \label{fig:agnes-average-full-ss}
\end{figure}

Na podstawie krzywych, przedstawionych na rysunku~\ref{fig:agnes-average-full-ss}, nie sposób jednoznacznie estymować optymalną liczbę grup. Z~racji zastosowanej metody łączenia, obliczone krzywe nie posiadają gładkiego charakteru, co zdecydowanie utrudnia ich analizę.

\begin{figure}[h]
    \centering
    \includegraphics[width=1.1\columnwidth]{./figures/full/agnes-average/sil.widths.pdf}
    \caption{Porównanie średniej szerokości sylwetki dla grupowania metodą aglomeracyjną z~połączeniem średnim dla różnych miar niepodobieństwa}
    \label{fig:agnes-average-full-sil}
\end{figure}

Na rysunku~\ref{fig:agnes-average-full-sil} porównane zostały przebiegi wartości średniej szerokości sylwetki w~zależności od  liczby grup. Krzywe: niebieska, czerwona oraz zielona przedstawiają podobny, mało zadowalający przebieg. Charakterystyka krzywej odpowiadającej mierze korelacyjnej ponownie posiada zupełnie inny charakter od reszty. Dla $k > 13$ wskaźnik jest większy od wszystkich pozostałych. 

\begin{figure}[h]
    \centering
    \includegraphics[width=1.1\columnwidth]{./figures/full/agnes-average/ch.pdf}
    \caption{Porównanie wskaźnika Callinskiego-Harabasza dla grupowania metodą aglomeracyjną z~połączeniem średnim dla różnych miar niepodobieństwa}
    \label{fig:agnes-average-full-ch}
\end{figure}

Rysunek~\ref{fig:agnes-average-full-ch} przedstawia zależność wskaźnika Callinskiego- Harabasza od liczby grup. Wraz z~rosnącą liczbą grup, wartość wskaźnika maleje. Najlepszą jakość grupowania osiągnięto dla odległości Manhattan, natomiast najgorszą dla miary korelacyjnej. Grupowanie przy użyciu miary Manhattan reaguje na podział \emph{ze względu na szczegółową pozycję}, na co wskazuje niewielki wzrost wskaźnika dla $k = 12$.
 
\begin{figure}[h]
    \centering
    \includegraphics[width=1.1\columnwidth]{./figures/full/agnes-average/dunn.pdf}
    \caption{Porównanie wskaźnika Dunna dla grupowania metodą aglomeracyjną z~połączeniem średnim dla różnych miar niepodobieństwa}
    \label{fig:agnes-average-full-dunn}
\end{figure}

Ostatnim badanym wskaźnikiem jakości wewnętrznej jest wskaźnik Dunna, którego zależność od liczby grup została przedstawiona na rysunku~\ref{fig:agnes-average-full-dunn}. Dla wszystkich czterech miar niepodobieństwa, wartość wskaźnika Dunna utrzymuje się na stałym, bardzo niskim poziomie, co wskazuje na niską jakość grupowania. Podobnie jak przy poprzednim sposobie łączenia, miara korelacyjna również osiąga najgorsze rezultaty.

Uzyskane wyniki wskazują jednoznacznie, że algorytm aglomeracyjny zarówno z~połączeniem kompletnym jak i~średnim, nie poradził sobie z~grupowaniem podanych danych. Najlepszy uzyskany podział udało się uzyskać dla parametru $k=2$. Podział taki dobrze opisuje pierwszy sposób opisu danych, czyli \emph{na podstawie gry w~polu}. Aby potwierdzić tą hipotezę, w~następnej kolejności zbadane zostały miary oceny zewnętrznej.

\begin{figure}[H]
    \centering
    \includegraphics[width=1.1\columnwidth]{./figures/full/agnes-average/rand-outfield.pdf}
    \caption{Porównanie skorygowanego wskaźnika Randa dla grupowania metodą aglomeracyjną z~połączeniem średnim dla różnych miar niepodobieństwa przy podziale na podstawie gry w~polu}
    \label{fig:agnes-average-full-rand-outfield}
\end{figure}


Aby sprawdzić jak dobrze uzyskane etykietowania opisują ten podział, zostanie zbadany skorygowany wskaźnik Randa, którego wartości wskaźnika zostały przedstawione na rysunku~\ref{fig:agnes-average-full-rand-outfield}. Dla takiego podziału, skorygowany wskaźnik Randa, uzyskuje najwyższą wartość dla miary bazującej na odległości Minkowskiego. Miara korelacyjna, podobnie jak przy miarach wewnętrznych, wypada najsłabiej, a~wartość indeksu oscyluje w~okolicach zera, osiągając dla $k < 3$ wartości ujemne, świadczące o~zupełnie niepoprawnym grupowaniu. 


Miara odległości współdzielonej informacji dla miary Minkowskiego osiąga wartość zerową, co oznacza stuprocentowe dopasowanie etykietowań. Zgodnie z~oczekiwaniami, wartość tej miary rośnie, wraz z~rosnącą liczbą grup. Pełna zależność dla wszystkich grup została przedstawiona na rysunku~\ref{fig:agnes-average-full-vi-outfield}. Rysunek ten dobrze ukazuje przepaść pomiędzy klasycznymi miarami a~miarą korelacyjną.

\begin{figure}[h]
    \centering
    \includegraphics[width=1.1\columnwidth]{./figures/full/agnes-average/vi-outfield.pdf}
    \caption{Porównanie odległości współdzielonej informacji dla grupowania metodą aglomeracyjną z~połączeniem średnim dla różnych miar niepodobieństwa przy podziale na podstawie gry w~polu}
    \label{fig:agnes-average-full-vi-outfield}
\end{figure}

\begin{figure}[h]
    \centering
    \includegraphics[width=1.1\columnwidth]{./figures/full/agnes-average/rand-general.pdf}
    \caption{Porównanie skorygowanego wskaźnika Randa dla grupowania metodą aglomeracyjną z~połączeniem średnim dla różnych miar niepodobieństwa przy podziale na podstawie ogólnej pozycji}
    \label{fig:agnes-average-full-rand-general}
\end{figure}


Dla innych podziałów, jakość grupowania nie jest już tak zadowalająca. Na rysunku~\ref{fig:agnes-average-full-rand-general} przedstawiono zależność skorygowanego indeksu Randa od liczby grup przy podziale na podstawie ogólnej pozycji. Miary klasyczne maleją wraz z~rosnącą liczbą grup, natomiast miara korelacyjna szybko wzrasta po przekroczeniu $k > 6$ i~utrzymuje względnie stałą wartość.


%%%%%%%%%%%%%%%%%%%%%%%%%%%%%%%%%%%%%%%%%%%%%%%%%%%%%%%%%%%%%%%%%%%%%%%%%%%%%%%%%%%%%%%%%%%%%%%%%%%%%%%%%%%%%%%%%%%%%%%%%%%%%%%%%%%%%%%%%%%%%%%%

\subsection{Algorytm deglomeracyjny}
\label{subsec:wyniki-diana}
\emph{Kod realizujący poniżej przedstawione eksperymenty został umieszczony w~pliku \texttt{diana-clustering.R}}.

Algorytm deglomeracyjny działa na podobnej zasadzie jak omawiany wcześniej algorytm deglomeracyjny. W~wyniku jego działania, również tworzona jest hierarchia przykładów, którą w~dalszej kolejności można dowolnie podzielić, otrzymując zadaną liczbę grup. W~przypadku tego algorytmu, nie ma potrzeby definiowania metody łączenia przykładów, ponieważ punktem startowym jest grupa zawierająca wszystkie przykłady, która następnie jest dzielona na podgrupy.

\begin{figure}[H]
    \centering
    \captionsetup[subfloat]{farskip=2pt,captionskip=1pt}
    
    \subfloat[Miara euklidesowa  \label{fig:diana-full-ss-euclidean}]{\includegraphics[width=0.5\columnwidth]{./figures/full/diana/ss.euclidean.pdf}}
    \subfloat[Miara manhattan \label{fig:diana-full-ss-manhattan}]{\includegraphics[width=0.5\columnwidth]{./figures/full/diana/ss.manhattan.pdf}}\hfill
    
    \subfloat[Miara Minkowskiego]{\includegraphics[width=0.5\columnwidth]{./figures/full/diana/ss.minkowski.pdf}}
    \subfloat[Miara korelacyjna \label{fig:diana-full-ss-correlation}]{\includegraphics[width=0.5\columnwidth]{./figures/full/diana/ss.correlation.pdf}}\hfill
    \caption{Porównanie krzywych sumy kwadratów odległości przy grupowaniu algorytmem deglomeracyjnym dla różnych miar niepodobieństwa}
    \label{fig:diana-full-ss}
\end{figure}

Na podstawie krzywych, przedstawionych na rysunku~\ref{fig:diana-full-ss}, można spróbować dokonać estymacji optymalnej liczby grup. Korzystając z~rysunków~\ref{fig:diana-full-ss-euclidean} oraz \ref{fig:diana-full-ss-manhattan} wnioskujemy, iż optymalna liczba grup wynosi od $2$ do $8$.

\begin{figure}[h]
    \centering
    \includegraphics[width=1.1\columnwidth]{./figures/full/diana/sil.widths.pdf}
    \caption{Porównanie średniej szerokości sylwetki dla grupowania algorytmem deglomeracyjnym dla różnych miar niepodobieństwa}
    \label{fig:diana-full-sil}
\end{figure}

Na rysunku~\ref{fig:diana-full-sil} porównane zostały przebiegi wartości średniej szerokości sylwetki w~zależności od zadanej liczby grup. Krzywe: niebieska, czerwona oraz zielona przedstawiają podobny przebieg. Najwyższą wartość uzyskują dla $k=2$, jednak dla $k < 15$, utrzymuje się dużo większa wartość niż dla pozostałych etykietowań. Charakterystyka krzywej odpowiadającej mierze korelacyjnej ponownie posiada zupełnie inny charakter od reszty. Dla $k > 15$, miara ta uzyskuje najlepsze wartości z~wszystkich pozostałych.

\begin{figure}[H]
    \centering
    \includegraphics[width=1.1\columnwidth]{./figures/full/diana/ch.pdf}
    \caption{Porównanie wskaźnika Callinskiego-Harabasza dla grupowania algorytmem deglomeracyjnym dla różnych miar niepodobieństwa}
    \label{fig:diana-full-ch}
\end{figure}

Rysunek~\ref{fig:diana-full-ch} przedstawia zależność wskaźnika Callinskiego-Harabasza w~zależności od liczby grup. Wraz z~rosnącą liczbą grup, wartość wskaźnika maleje.  Z~rysunku wynika również, że najlepszą jakość grupowania osiągnięto dla miary bazującej na odległości Manhattan, natomiast najgorszą dla miary korelacyjnej.
 
\begin{figure}[h]
    \centering
    \includegraphics[width=1.1\columnwidth]{./figures/full/diana/dunn.pdf}
    \caption{Porównanie wskaźnika Dunna dla grupowania algorytmem deglomeracyjnym dla różnych miar niepodobieństwa}
    \label{fig:diana-full-dunn}
\end{figure}

Ostatnim badanym wskaźnikiem jakości wewnętrznej jest wskaźnik Dunna, którego zależność od liczby grup została przedstawiona na rysunku~\ref{fig:diana-full-dunn}. Dla wszystkich czterech miar niepodobieństwa, wartość wskaźnika Dunna utrzymuje się na stałym, bardzo niskim poziomie, co wskazuje na niską jakość grupowania. Podobnie jak we wcześniejszych przypadkach, miara korelacyjna również osiąga najgorsze rezultaty.

Uzyskane wyniki wskazują jednoznacznie, że algorytm deglomeracyjny również nie poradził sobie z~grupowaniem podanych danych. Najlepszy uzyskany podział udało się uzyskać dla parametru $k=2$. Podział taki dobrze opisuje pierwszy sposób opisu danych, czyli \emph{na podstawie gry w~polu}. Aby sprawdzić jak dobrze uzyskane etykietowania opisują ten podział, zostanie zbadany skorygowany wskaźnik Randa, którego wartości wskaźnika zostały przedstawione na rysunku~\ref{fig:diana-full-rand-outfield}. 

Dla takiego podziału, skorygowany wskaźnik Randa, uzyskuje najwyższą wartość dla miar klasycznych. Dla tych miar, grupowanie, mimo niskiej jakości wewnętrznej, poprawnie sklasyfikowało wszystkie przykłady ze zbioru. Miara korelacyjna, podobnie jak przy miarach wewnętrznych, wypada najsłabiej, a~wartość indeksu oscyluje w~okolicach zera.

\begin{figure}[h]
    \centering
    \includegraphics[width=1.1\columnwidth]{./figures/full/diana/rand-outfield.pdf}
    \caption{Porównanie skorygowanego wskaźnika Randa dla grupowania algorytmem deglomeracyjnym dla różnych miar niepodobieństwa przy podziale na podstawie gry w~polu}
    \label{fig:diana-full-rand-outfield}
\end{figure}

Miara odległości współdzielonej informacji dla miary Manhattan osiąga wartość zerową, co oznacza stuprocentowe dopasowanie etykietowań. Zgodnie z~oczekiwaniami, wartość tej miary rośnie, wraz z~rosnącą liczbą grup. Pełna zależność dla wszystkich grup została przedstawiona na rysunku~\ref{fig:diana-full-vi-outfield}.

\begin{figure}[h]
    \centering
    \includegraphics[width=1.1\columnwidth]{./figures/full/diana/vi-outfield.pdf}
    \caption{Porównanie odległości współdzielonej informacji dla grupowania algorytmem deglomeracyjnym dla różnych miar niepodobieństwa przy podziale na podstawie gry w~polu}
    \label{fig:diana-full-vi-outfield}
\end{figure}

%%%%%%%%%%%%%%%%%%%%%%%%%%%%%%%%%%%%%%%%%%%%%%%%%%%%%%%%%%%%%%%%%%%%%%%%%%%%%%%%%%%%%%%%%%%%%%%%%%%%%%%%%%%%%%%%%%%%%%%%%%%%%%%%%%%%%%%%%%%%%%%%%%%

\subsection{Algorytm FANNY}
\label{subsec:wyniki-fanny}
\emph{Kod realizujący poniżej przedstawione eksperymenty został umieszczony w~pliku \texttt{fanny-clustering.R}}.

Dla grupowań rozmytego algorytmem FANNY, zdecydowaliśmy się na policzenie tylko jednego wskaźnika, którym jest rozmyty średni wskaźnik sylwetki. Porównanie wartości tego wskaźnika dla różnej liczby zadanych grup zostało przedstawione na rysunku~\ref{fig:fanny-full-fsil}. 

\begin{figure}[h]
    \centering
    \includegraphics[width=1.1\columnwidth]{./figures/full/fanny/fsil.pdf}
    \caption{Porównanie rozmytego średniego wskaźnika sylwetki dla grupowania algorytmem FANNY dla różnych miar niepodobieństwa}
    \label{fig:fanny-full-fsil}
\end{figure}

Na podstawie rysunku, możliwym jest określenie która z~miar niepodobieństwa jest najlepsza do danego zadania. Dla niskich wartości parametru $k$ przodują miary klasyczne, czyli: euklidesowa, Manhattan oraz Minkowskiego. Dzieje się tak ponieważ dla podziału \emph{na podstawie gry w~polu}, rozmycie funkcji przynależności jest niewielkie. Dla większych wartości parametru $k$, miara bazująca na korelacji zaczyna przeważać nad pozostałymi. Zakres, w~którym miara ta przeważa nad konkurentami, odpowiada mniej więcej podziałowi \emph{na podstawie szczegółowej pozycji}.

%%%%%%%%%%%%%%%%%%%%%%%%%%%%%%%%%%%%%%%%%%%%%%%%%%%%%%%%%%%%%%%%%%%%%%%%%%%%%%%%%%%%%%%%%%%%%%%%%%%%%%%%%%%%%%%%%%%%%%%%%%%%%%%%%%%%%%%%%%%%

\subsection{Rozmyte k-średnich}
\label{subsec:wyniki-fcm}
\emph{Kod realizujący poniżej przedstawione eksperymenty został umieszczony w~pliku \texttt{cmeans-clustering.R}}.

Dla grupowania algorytmem rozmytych k-średnich, ponownie zdecydowaliśmy się na policzenie tylko jednego wskaźnika, którym jest rozmyty wskaźnik średniej sylwetki. Porównanie wartości tego wskaźnika dla różnej liczby zadanych grup zostało przedstawione na rysunku~\ref{fig:fanny-full-fsil}. 

\begin{figure}[h]
    \centering
    \includegraphics[width=1.1\columnwidth]{./figures/full/cmeans/fsil.pdf}
    \caption{Porównanie rozmytego średniego wskaźnika sylwetki dla grupowania metodą rozmytych k-średnich dla różnych miar niepodobieństwa}
    \label{fig:fanny-full-fsil}
\end{figure}

W~przeciwieństwie do algorytmu FANNY, rozmyty wskaźnik średniej sylwetki dla algorytmu rozmytych k-średnich ma mocno oscylacyjny charakter, który wynika z~faktu braku jawnej optymalizacji wskaźnika jakości, jak to się dzieje w~przypadku poprzedniego algorytmu. W~ogólności, najwyższe wartości tego wskaźnika udało się uzyskać dla $k < 20$, które zgrubnie można przyporządkować granicy zasadności stosowania podziału \emph{na podstawie szczegółowej pozycji}.

\section{Możliwości poprawy}
\label{subsec:poprawa}

Analizując uzyskane wyniki, doszliśmy do wniosku, że żaden z~badanych algorytmów nie poradził sobie dostatecznie z~postawionym zadaniem. W~tym celu, podjęliśmy dwie próby poprawy jakości etykietowania. W~pierwszym podejściu, ograniczyliśmy przestrzeń atrybutów, a~w~drugim zmieniliśmy podejście grupowania z~korzystającego z~niepodobieństwa, na korzystające z~gęstości przykładów.

\subsection{Dalsze ograniczenie zbioru atrybutów}
\label{subsec:dalsze-ograniczenie-liczby-atrybutow}

Po przeprowadzeniu wszystkich eksperymentów na zbiorze danych z liczbą atrybutów równą $20$, postanowiliśmy zbadać wyniki grupowania przy dalszym ograniczeniu liczby atrybutów. Na podstawie macierzy korelacji ustaliliśmy ostatecznie ograniczony zbiór, składający się z czterech atrybutów:

\begin{itemize}
    \item \emph{GK.Skills} -- umiejętności bramkarskie,
    \item \emph{Tackling} -- przejmowanie piłki,
    \item \emph{Short.Ball.Skills} -- gra na małym obszarze,
    \item \emph{Shooting} -- wykończenie.
\end{itemize}

Dla danych o takim zbiorze cech zostały powtórzone wszystkie wcześniej wykonywane eksperymenty.
Na opisywanych poniżej wykresach pokazane zostaną wartości miar jakości w zależności od wynikowej liczby grup zwracanych przez algorytmy. Dla każdego eksperymentu, ponownie obliczyliśmy wszystkie wskaźniki dla wszystkich miar niepodobieństwa.

%%%%%%%%%%%%%%%%%%%%%%%%%%%%%%%%%%%%%%%%%%%%%%%%%%%%%%%%%%%%%%%%%%%%%%%%%%%%%%%%%%%%%%%%%%%%%%%%%%%%%%%%%%%%%%%%%%%%%%%%%%%%%%%%%%%%%%%%%%%%%%%%%%%%%%%%%%%%%%%%%%%%%%%%%%%%%%%%%%%%%%%%%%%%%%%%%%%%%%%%%%%%%%%%%%%%%%%%%%%%

\subsubsection{Algorytm k-medoidów}
\label{subsec:poprawa-kmedoids}

Powtórzone eksperymenty dla mniejszej liczby atrybutów przyniosły nieco inne wyniki. Rysunek \ref{fig:kmed-short-rand-gen} przedstawia wykres wskaźnika Randa w zależności od wartości $k$. Rysunek \ref{fig:kmed-short-vi-gen} z kolei przedstawia wykresy wskaźnika współdzielonej informacji.

\begin{figure}[ht]
    \centering
    \includegraphics[width=1.1\columnwidth]{./figures/short/kmed/rand-general.pdf}
    \caption{Wskaźnik Randa dla grupowania metodą k-medoidów dla analizy na podstawie ogólnej pozycji}
    \label{fig:kmed-short-rand-gen}
\end{figure}

\begin{figure}[ht]
    \centering
    \includegraphics[width=1.1\columnwidth]{./figures/short/kmed/vi-general.pdf}
    \caption{Wskaźnik współdzielonej informacji dla grupowania metodą k-medoidów dla analizy na podstawie ogólnej pozycji}
    \label{fig:kmed-short-vi-gen}
\end{figure}

Można zauważyć, że najlepsze wartości wskaźnika otrzymano dla każdej z~metryk dla podziału na 3-4 grupy. W~przypadku odległości informacji współdzielonej, dla klasycznych miar odległości otrzymaliśmy podobne wyniki. Najmniejsze wartości indeks osiąga dla liczby $k=3$. Nie jest to idealny wynik, ponieważ w rozwiązaniu referencyjnym mamy cztery grupy odzwierciedlające podział na cztery ogólne pozycje.  

Dla algorytmu k-medoidów wewnętrzne miary jakości przedstawione zostały na wykresach \ref{fig:kmed-short-sil}, \ref{fig:kmed-short-ch} oraz \ref{fig:kmed-short-dunn}. 

Wykres średniej szerokości sylwetki znacząco różni się od swojego odpowiednika na rysunku \ref{fig:kmed-full-sil}. Wartości dla klasycznych miar odległości, podobnie jak wcześniej, maleją do pewnego poziomu, wraz ze zwiększaniem się liczby $k$. Po zmniejszeniu liczby atrybutów zupełnie inaczej wygląda wynik dla odległości korelacyjnej - uzyskujemy największe wartości wskaźnika jakości.

\begin{figure}[ht]
    \centering
    \includegraphics[width=1.1\columnwidth]{./figures/short/kmed/sil.widths.pdf}
    \caption{Porównanie średniej szerokości sylwetki dla grupowania metodą k-medoidów dla różnych miar niepodobieństwa}
    \label{fig:kmed-short-sil}
\end{figure}

Na rysunku \ref{fig:kmed-short-ch} obserwujemy wartości wskaźnika Callinskiego-Harabasza. Dla klasycznych metryk, wykres wygląda podobnie jak przy wykorzystaniu pełnego zbioru atrybutów, z tą różnicą, że maksymalne wartości osiągamy dla podziału na cztery grupy. Możemy wnioskować, że uzyskaliśmy poprawę jakości grupowania. Dla miary korelacyjnej, wartość wskaźnika rośnie wraz ze wzrostem liczby grup.

\begin{figure}[ht]
    \centering
    \includegraphics[width=1.1\columnwidth]{./figures/short/kmed/ch.pdf}
    \caption{Porównanie wskaźnika Callinskiego-Harabasza dla grupowania metodą k-medoidów dla różnych miar niepodobieństwaą}
    \label{fig:kmed-short-ch}
\end{figure}

Wykresy wskaźnika Dunna przedstawione na rysunku \ref{fig:kmed-short-dunn} pokazują z kolei, że najlepsze wartości wskaźnika uzyskujemy dla podziału na dwie grupy. Dla pozostałych wartości $k$ otrzymujemy bardzo niskie wyniki co świadczy o niskiej jakości grupowania.

\begin{figure}[ht]
    \centering
    \includegraphics[width=1.1\columnwidth]{./figures/short/kmed/dunn.pdf}
    \caption{Porównanie wskaźnika Dunna dla grupowania metodą k-medoidów dla różnych miar niepodobieństwa}
    \label{fig:kmed-short-dunn}
\end{figure}

Zmniejszenie zbioru atrybutów pozwoliło na uzyskanie znacznie lepszych wyników, od tych otrzymanych przy wykorzystaniu pełnego zbioru atrybutów. Mniejszy zbiór atrybutów pozwolił algorytmom bazującym na mierze korelacyjnej, lepiej odnaleźć zależności pomiędzy zawodnikami, co poskutkowało polepszeniem się wskaźników jakości wewnętrznej.

%%%%%%%%%%%%%%%%%%%%%%%%%%%%%%%%%%%%%%%%%%%%%%%%%%%%%%%%%%%%%%%%%%%%%%%%%%%%%%%%%%%%%%%%%%%%%%%%%%%%%%%%%%%%%%%%%%%%%%%%%%%%%%%%%%%%%%%%%%%%%%%%%%%%%%%%%%%%%%%%%%%%%%%%%%%%%%%%%%%%%%%%%%%%%%%%%%%%%%%%%%%%%%%%%%%%%%%%%%%%

\subsubsection{Algorytm aglomeracyjny}
\label{subsec:poprawa-agnes}
Eksperymenty dla algorytmów aglomeracyjnych ponownie zostały przeprowadzone przy wykorzystaniu dwóch metod łączenia grup wspomnianych w \ref{subsec:hierarchical}. 

\subsubsection{Połączenie kompletne}

Zewnętrzne miary jakości dla połączenia kompletnego przedstawiono na wykresach \ref{fig:agnes-complete-short-rand-gen} i \ref{fig:agnes-complete-short-vi-gen}. Kolejny raz przedstawiamy wykresy miar dla analizy na podstawie ogólnej pozycji zawodnika. Wyniki otrzymane po przejściu hierarchicznego algorytmu aglomeracyjnego przedstawiają się podobnie jak dla algorytmu k-medoidów. Skorygowany indeks Randa znowu największe wartości uzyskał dla 3-4 grup wynikowych. Najgorsze wyniki otrzymaliśmy dla odległości korelacyjnej. 

\begin{figure}[ht]
    \centering
    \includegraphics[width=1.1\columnwidth]{./figures/short/agnes-complete/rand-general.pdf}
    \caption{Wskaźnik Randa dla grupowania metodą aglomeracyjną dla analizy na podstawie ogólnej pozycji}
    \label{fig:agnes-complete-short-rand-gen}
\end{figure}

Wykres wskaźnika współdzielonej informacji nieco różni się od wykresu otrzymanego dla algorytmu k-medoidów. Wyraźnie widać, że dla klasycznych metryk znów najniższe wartości uzyskujemy dla $k=3$. Po spojrzeniu na wykres widzimy również, że najlepszy wynik otrzymaliśmy dla odległości Minkowskiego. 

\begin{figure}[ht]
    \centering
    \includegraphics[width=1.1\columnwidth]{./figures/short/agnes-complete/vi-general.pdf}
    \caption{Wskaźnik współdzielonej informacji dla grupowania metodą aglomeracyjną dla analizy na podstawie ogólnej pozycji}
    \label{fig:agnes-complete-short-vi-gen}
\end{figure}

Wskaźnik szerokości sylwetki, przedstawiony na rysunku~\ref{fig:agnes-complete-short-sil}, prezentuje się podobne jak w~przypadku~\ref{fig:kmed-short-sil}. Najwyższe wyniki otrzymaliśmy dla odległości korelacyjnej, jednak ogólna jakość grupowania przy użyciu metryk klasycznych znów nie była najlepsza. Algorytm uzyskał najwyższe wartości wskaźnika szerokości sylwetki dla $k=2$.   

\begin{figure}[ht]
    \centering
    \includegraphics[width=1.1\columnwidth]{./figures/short/agnes-complete/sil.widths.pdf}
    \caption{Porównanie średniej szerokości sylwetki dla grupowania metodą aglomeracyjną z~połączeniem kompletnym dla różnych miar niepodobieństwa}
    \label{fig:agnes-complete-short-sil}
\end{figure}

Patrząc na rysunek \ref{fig:agnes-complete-short-ch} obserwujemy wykresy wskaźnika Callinskiego-Harabasza dla grupowania otrzymanego algorytmem aglomeracyjnym. Na podstawie tego wykresu ciężko wskazać optymalną liczbę grup. Dla miar klasycznych prawdopodobnie byłaby to liczba $k=4$ (euklidesowa, Minkowski) lub $k=5$ (Manhattan). Wartość dla odległości korelacyjnej znów rośnie w miarę zwiększania się liczby grup i maksymalną wartość uzyskuje dla $k=40$.

\begin{figure}[H]
    \centering
    \includegraphics[width=1.1\columnwidth]{./figures/short/agnes-complete/ch.pdf}
    \caption{Porównanie wskaźnika Callinskiego-Harabasza dla grupowania metodą aglomeracyjną z~połączeniem kompletnym dla różnych miar niepodobieństwa}
    \label{fig:agnes-complete-short-ch}
\end{figure}

Ostatnim wyliczanym wskaźnikiem jakości wewnętrznej ponownie był wskaźnik Dunna. Jego zależność od liczby grup została przedstawiona na rysunku~\ref{fig:agnes-complete-short-dunn}. Dla wszystkich czterech miar niepodobieństwa, przez niemal cały eksperyment, wartość wskaźnika Dunna utrzymuje się na stałym, bardzo niskim poziomie, co wskazuje na niską jakość grupowania. Jedyne dość wysokie wyniki otrzymano dla klasycznych miar odległości. Podobnie jak przy algorytmie k-medoidów, miara korelacyjna również osiąga najgorsze rezultaty.

\begin{figure}[ht]
    \centering
    \includegraphics[width=1.1\columnwidth]{./figures/short/agnes-complete/dunn.pdf}
    \caption{Porównanie wskaźnika Dunna dla grupowania metodą aglomeracyjną z~połączeniem kompletnym dla różnych miar niepodobieństwa}
    \label{fig:agnes-complete-short-dunn}
\end{figure}

%%%%%%%%%%%%%%%%%%%%%%%%%%%%%%%%%%%%%%%%%%%%%%%%%%%%%%%%%%%%%%%%%%%%%%%%%%%%%%%%%%%%%%%%%%%%%%%%%%%%%%%%%%%%%%%%%%%%%%%%%%%%%%%%%%%%%%%%%%%%%%%%%%%%%%%%%%%%%%%%%%%%%%%%%%%%%%%%%%%%%%%%%%%%%%%%%%%%%%%%%%%%%%%%%%%%%%%%%%%%

\subsubsection{Połączenie średnie}

Dla sposobu łączenia w grupy zwanego połączeniem średnim powtórzono eksperymenty, a ich wyniki przedstawiono poniżej. 

\begin{figure}[H]
    \centering
    \includegraphics[width=1.1\columnwidth]{./figures/short/agnes-average/rand-general.pdf}
    \caption{Wskaźnik Randa dla grupowania metodą aglomeracyjną dla analizy na podstawie ogólnej pozycji}
    \label{fig:agnes-average-short-rand-gen}
\end{figure}

Miary zewnętrzne wyliczane na podstawie wyników grupowania zostały przedstawione na rysunkach \ref{fig:agnes-average-short-rand-gen} i \ref{fig:agnes-average-short-vi-gen}. Uzyskane wyniki wskazują jednoznacznie, że algorytm aglomeracyjny z~połączeniem średnim nie poradził sobie z~grupowaniem podanych danych. Patrząc na rysunek \ref{fig:agnes-average-short-rand-gen} można stwierdzić, że optymalna liczba grup leży między $k=3$ a $k=5$. Jest to wynik poprawny zważywszy na fakt że zbiór referencyjny jest podzielony na cztery grupy. Dla takiego podziału, skorygowany wskaźnik Randa, uzyskuje zbliżone wartości dla metryk klasycznych. Miara korelacyjna, podobnie jak przy poprzednich eksperymentach, wypada najsłabiej.

\begin{figure}[ht]
    \centering
    \includegraphics[width=1.1\columnwidth]{./figures/short/agnes-average/vi-general.pdf}
    \caption{Wskaźnik współdzielonej informacji dla grupowania metodą aglomeracyjną dla analizy na podstawie ogólnej pozycji}
    \label{fig:agnes-average-short-vi-gen}
\end{figure}

Miara odległości współdzielonej informacji dla miary Minkowskiego osiąga wartość najmniejszą, co oznacza najlepsze dopasowanie etykietowań. Tak jak we wcześniejszych eksperymentach oraz zgodnie z~oczekiwaniami, wartość tej miary rośnie, wraz z~rosnącą liczbą grup.

\begin{figure}[H]
    \centering
    \includegraphics[width=1.1\columnwidth]{./figures/short/agnes-average/sil.widths.pdf}
    \caption{Porównanie średniej szerokości sylwetki dla grupowania metodą aglomeracyjną z~połaczeniem średnim dla różnych miar niepodobieństwa}
    \label{fig:agnes-average-short-sil}
\end{figure}

Na rysunku~\ref{fig:agnes-average-short-sil} porównane zostały przebiegi wartości średniej szerokości sylwetki w~zależności od zadanej liczby grup. Krzywe: niebieska, czerwona oraz zielona przedstawiają podobny, mało zadowalający przebieg. Charakterystyka krzywej odpowiadającej posiada inny charakter od reszty. Zmiana parametru $k$ sprawia że wartość oscyluje między $\num{0,4}$ a $\num{0,5}$. Ponadto dla $k$ większych od 8 miara wskaźnika dla miary korelacyjnej jest zdecydowanie większa od wskaźników dla pozostałych miar.

\begin{figure}[ht]
    \centering
    \includegraphics[width=1.1\columnwidth]{./figures/short/agnes-average/ch.pdf}
    \caption{Porównanie wskaźnika Callinskiego-Harabasza dla grupowania metodą aglomeracyjną z~połączeniem średnim dla różnych miar niepodobieństwa}
    \label{fig:agnes-average-short-ch}
\end{figure}

Na rysunku \ref{fig:agnes-average-short-ch} obserwujemy wartości wskaźnika Callinskiego-Harabasza. Dla wszystkich metryk niepodobieństwa, wykres wygląda podobnie jak przy wykorzystaniu połączenia kompletnego, co może wskazywać na niewrażliwość tego wskaźnika na metodę łączenia.

\begin{figure}[H]
    \centering
    \includegraphics[width=1.1\columnwidth]{./figures/short/agnes-average/dunn.pdf}
    \caption{Porównanie wskaźnika Dunna dla grupowania metodą aglomeracyjną z~połączeniem średnim dla różnych miar niepodobieństwa}
    \label{fig:agnes-average-short-dunn}
\end{figure}


W przypadku wskaźnika Dunna obserwujemy niemal identyczny wynik jak w przypadku poprzedzających algorytmów. Wartość dla miary korelacyjnej przez cały eksperyment jest bliska zeru. Dla pozostałych metryk obserwujemy niemal skokowy spadek z dość wysokich wartości praktycznie do zera przy przejściu z $k=2$ na $k=3$.  


%%%%%%%%%%%%%%%%%%%%%%%%%%%%%%%%%%%%%%%%%%%%%%%%%%%%%%%%%%%%%%%%%%%%%%%%%%%%%%%%%%%%%%%%%%%%%%%%%%%%%%%%%%%%%%%%%%%%%%%%%%%%%%%%%%%%%%%%%%%%%%%%%%%%%%%%%%%%%%%%%%%%%%%%%%%%%%%%%%%%%%%%%%%%%%%%%%%%%%%%%%%%%%%%%%%%%%%%%%%%

\subsubsection{Algorytm deglomeracyjny}
\label{subsec:wyniki-diana}
Na rysunku \ref{fig:diana-short-rand-gen} możemy zobaczyć, że najlepszym wynikiem indeksu Randa cechuje się podział na trzy grupy. W tym punkcie $k$ otrzymaliśmy maksymalne wartości dla każdej metryki. Dla $k>3$ znowu obserwujemy spadek wartości wskaźnika. 

Wykresy wskaźnika współdzielonej informacji wyglądają bardzo podobnie do odpowiadających im wykresom przedstawianym wcześniej. W szczególności jest on podobny do wykresu otrzymanego dla algorytmu aglomeracyjnego z połączeniem kompletnym gdzie wykresy dla metryk klasycznych niemal się ze sobą pokrywały. Ponownie najlepszy wynik uzyskujemy dla odległości Minkowskiego dla $k=3$.

\begin{figure}[H]
    \centering
    \includegraphics[width=1.1\columnwidth]{./figures/short/diana/rand-general.pdf}
    \caption{Wskaźnik Randa dla grupowania metodą deglomeracyjną dla analizy na podstawie ogólnej pozycji}
    \label{fig:diana-short-rand-gen}
\end{figure}

Zależność wskaźnika odległości współdzielonej informacji od liczby grup z~rysunku~\ref{fig:diana-short-vi-gen}, jest bliźniacza do przedstawionych wcześniej zależności przedstawionych w~poprzedniej części sprawozdania. Z~analizy wykresu możemy wywnioskować że grupowanie przy użyciu miary korelacyjnej daje najgorsze wyniki w~kategorii miar zewnętrznych. 

\begin{figure}[ht]
    \centering
    \includegraphics[width=1.1\columnwidth]{./figures/short/diana/vi-general.pdf}
    \caption{Wskaźnik współdzielonej informacji dla grupowania metodą deglomeracyjną dla analizy na podstawie ogólnej pozycji}
    \label{fig:diana-short-vi-gen}
\end{figure}

Na rysunku~\ref{fig:diana-short-sil} porównane zostały przebiegi wartości średniej szerokości sylwetki. Wskaźnik dla miar klasycznych posiada dobrze już znany, malejący przebieg. Charakterystyka krzywej odpowiadającej mierze korelacyjnej kolejny raz posiada zupełnie inny charakter od reszty. Zmiana parametru $k$ poraz kolejny nie wpływa zbytnio na wzrost czy spadek wskaźnika. 

\begin{figure}[H]
    \centering
    \includegraphics[width=1.1\columnwidth]{./figures/short/diana/sil.widths.pdf}
    \caption{Porównanie średniej szerokości sylwetki dla grupowania metodą deglomeracyjną  dla różnych miar niepodobieństwa}
    \label{fig:diana-short-sil}
\end{figure}

Patrząc na rysunek \ref{fig:diana-short-ch} obserwujemy wykresy wskaźnika Callinskiego-Harabasza dla algorytmu deglomeracyjnego. Na podstawie tego wykresu ciężko wskazać optymalną liczbę grup. Dla miar klasycznych prawdopodobnie byłaby to liczba z przedziału $[3,4]$. Wartość dla odległości korelacyjnej znów rośnie od początku do końca trwania eksperymentu.

\begin{figure}[ht]
    \centering
    \includegraphics[width=1.1\columnwidth]{./figures/short/diana/ch.pdf}
    \caption{Porównanie wskaźnika Callinskiego-Harabasza dla grupowania metodą deglomeracyjną dla różnych miar niepodobieństwa}
    \label{fig:diana-short-ch}
\end{figure}

Tak jak przypuszczaliśmy, otrzymaliśmy ponownie mało zadowalające wykresy dla wskaźnika Dunna, który przez większość eksperymentu jest bliski zeru z wyjątkiem pierwszego przejścia algorytmu dla dwóch grup wynikowych przy wykorzystaniu miar klasycznych.

\begin{figure}[h]
    \centering
    \includegraphics[width=1.1\columnwidth]{./figures/short/diana/dunn.pdf}
    \caption{Porównanie wskaźnika Dunna dla grupowania metodą deglomeracyjną dla różnych miar niepodobieństwa}
    \label{fig:diana-short-dunn}
\end{figure}

Dla obu typów algorytmów hierarchicznych, uzyskaliśmy poprawę jakości wewnętrznej uzyskanych etykietowań. Szczególny wzrost został zaobserwowany dla algorytmów bazujących na mierze korelacyjnej. Mniejsza liczba atrybutów pozwoliła na lepsze rozpoznawanie wzorców w~danych.

%%%%%%%%%%%%%%%%%%%%%%%%%%%%%%%%%%%%%%%%%%%%%%%%%%%%%%%%%%%%%%%%%%%%%%%%%%%%%%%%%%%%%%%%%%%%%%%%%%%%%%%%%%%%%%%%%%%%%%%%%%%%%%%%%%%%%%%%%%%%%%%%%%%%%%%%%%%%%%%%%%%%%%%%%%%%%%%%%%%%%%%%%%%%%%%%%%%%%%%%%%%%%%%%%%%%%%%%%%%%

\subsubsection{Algorytm FANNY}
\label{subsec:poprawa-fanny}

Porównanie wartości rozmytego średniego wskaźnika sylwetki dla różnej liczby zadanych grup zostało przedstawione na rysunku~\ref{fig:fanny-short-fsil}. 

\begin{figure}[ht]
    \centering
    \includegraphics[width=1.1\columnwidth]{./figures/short/fanny/fsil.pdf}
    \caption{Porównanie rozmytego średniego wskaźnika sylwetki dla grupowania algorytmem FANNY dla różnych miar niepodobieństwa}
    \label{fig:fanny-short-fsil}
\end{figure}

 Tak jak dla eksperymentu z pełnym zbiorem atrybutów, niskich wartości parametru $k$ przodują miary klasyczne, czyli: euklidesowa, manhattan oraz Minkowskiego. W końcowym etapie eksperymentu wskaźniki dla miar klasycznych mają wyraźnie oscylacyjny przebieg, co nie miało miejsca dla danych z pełnym zbiorem atrybutów. Dla większych wartości parametru $k$, miara bazująca na korelacji kolejny raz zaczyna przeważać nad pozostałymi. Zakres, w~którym miara ta przeważa nad konkurentami, odpowiada mniej więcej podziałowi \emph{na podstawie szczegółowej pozycji}. Przebieg dla tej miary ma również najspokojniejszy przebieg na przestrzeni całego eksperymentu.

%%%%%%%%%%%%%%%%%%%%%%%%%%%%%%%%%%%%%%%%%%%%%%%%%%%%%%%%%%%%%%%%%%%%%%%%%%%%%%%%%%%%%%%%%%%%%%%%%%%%%%%%%%%%%%%%%%%%%%%%%%%%%%%%%%%%%%%%%%%%%%%%%%%%%%%%%%%%%%%%%%%%%%%%%%%%%%%%%%%%%%%%%%%%%%%%%%%%%%%%%%%%%%%%%%%%%%%%%%%%

\subsubsection{Rozmyte k-średnich}
\label{subsec:wyniki-fcm}

Tak jak dla przypadku z pełnym zbiorem atrybutów, policzyliśmy tylko jeden wskaźnik - rozmyty średni wskaźnik sylwetki. Porównanie wartości tego wskaźnika dla różnej liczby zadanych grup zostało przedstawione na rysunku~\ref{fig:cmeans-short-fsil}. 

\begin{figure}[ht]
    \centering
    \includegraphics[width=1.1\columnwidth]{./figures/short/cmeans/fsil.pdf}
    \caption{Porównanie rozmytego średniego wskaźnika sylwetki dla grupowania metodą fuzzy c-means dla różnych miar niepodobieństwa}
    \label{fig:cmeans-short-fsil}
\end{figure}

Ponownie obserwujemy mocno oscylacyjny charakter przebiegów wskaźnika, wynikający z~faktu braku jawnej optymalizacji. Oscylacje jednak wydają się być mniejsze niż w przypadku eksperymentu z pełnym zbiorem atrybutów. Podobnie jak wcześniej, najwyższe wartości wskaźnika udało się uzyskać dla $k < 20$. Średnia wartość wskaźnika dla miary korelacyjnej, wskazuje na relatywnie dobrą jakość grupowania  

%%%%%%%%%%%%%%%%%%%%%%%%%%%%%%%%%%%%%%%%%%%%%%%%%%%%%%%%%%%%%%%%%%%%%%%%%%%%%%%%%%%%%%%%%%%%%%%%%%%%%%%%%%%%%%%%%%%%%%%%%%%%%%%%%%%%%%%%%%%%%%%%%%%%%%%%%%%%%%%%%%%%%%%%%%%%%%%%%%%%%%%%%%%%%%%%%%%%%%%%%%%%%%%%%%%%%%%%%%%%

\subsection{Automatyczny dobór liczby grup}
\label{subsec:automatyczny-dobor}

Grupowanie gęstościowe jest gałęzią algorytmów grupowania automatycznego, która jako grupy oznacza obszary w~przestrzeni atrybutów o~wysokiej gęstości przykładów. Dzięki takiemu podejściu, nie jest wymagane podawanie zadanej liczby grup \emph{a priori}. Algorytmy z~tej rodziny są w~stanie wykrywać grupy o~arbitralnym kształcie, dzięki czemu mogą wykrywać \emph{naturalne} grupy. Algorytmy te nadają się do przeprowadzenia grupowania na dużych zbiorach danych, o~których nie mamy żadnej innej wiedzy, dzięki czemu są chętnie wykorzystywane w~zagadnieniach związanych z~automatycznym odkrywaniem wiedzy. 

Najbardziej podstawowym algorytmem grupowania gęstościowego jest algorytm DBSCAN (\emph{Density-Based Spatial CLustering of Applications with Noise} \cite{edami}. Algorytm wyznacza punkty rdzeniowe, czyli takie które z~zadanym promieniu sąsiedztwa $\varepsilon$, mają co najmniej $M$ sąsiadów. Punkty rdzeniowe wyznaczają grupy, do których dołączane są punkty sąsiednie oraz inne grupy. Takie podejście pozwala na utworzenie dowolnej liczby grup, a~także wyznaczenie punktów które nie należą do żadnej grupy (tzw. punkty szumu).

Aby zbadać użyteczność algorytmu DBSCAN, wykorzystaliśmy ograniczony zbiór danych do $10\%$ pierwotnego zbioru, z~ograniczoną przestrzenią atrybutów do czterech (podobnie jak w~\ref{subsec:dalsze-ograniczenie-liczby-atrybutow}). W~trakcie eksperymentów zbadaliśmy przestrzeń parametrów $\varepsilon$ oraz $M$.

\begin{figure}[ht]
    \centering
    \includegraphics[width=1.1\columnwidth]{./figures/dbscan/cluster.pdf}
    \caption{Liczba wyznaczonych grup w~zależności od minimalnej liczby sąsiadów oraz maksymalnego promienia sąsiedztwa}
    \label{fig:dbscan-cluster}
\end{figure}

Na rysunku~\ref{fig:dbscan-cluster} przedstawiona została zależność znalezionej liczby grup od maksymalnej odległości $\varepsilon$ oraz minimalnej liczby punktów $M$. Najwięcej grup zostało odkrytych dla $M = 5$ oraz dla $\varepsilon = \num{0.5}$. Niska wartość parametru $M$ ułatwia przykładom łączenie się w~grupy. Dla niskich wartości $\varepsilon$, ciężej znaleźć wymaganą liczbę sąsiadów, natomiast dla wysokich wartości $\varepsilon$ tworzone jest mniej grup ale o~większej średniej liczbie przykładów. W~patologicznych sytuacjach, wysoka średnia wielkość grupy jest spowodowana tylko jedną wyznaczoną grupą. Zjawisko to zostało przedstawione na rysunku~\ref{fig:dbscan-avgsize}.

\begin{figure}[ht]
    \centering
    \includegraphics[width=1.1\columnwidth]{./figures/dbscan/avgsize.pdf}
    \caption{Średnia liczba punktów grupie w~zależności od minimalnej liczby sąsiadów oraz maksymalnego promienia sąsiedztwa}
    \label{fig:dbscan-avgsize}
\end{figure}

Wraz z~poluzowaniem ograniczeń związanych z~minimalną liczbą sąsiadów oraz maksymalną odległością pomiędzy punktami, rośnie procent punktów które znalazły przyporządkowanie do grupy. Na~rysunku~\ref{fig:dbscan-noise} zaprezentowana została zależność pomiędzy stosunkiem zakwalifikowanych punktów do wszystkich przykładów a maksymalną odległością oraz minimalną liczbą sąsiadów. Dla $M = 5$ wartość ta rośnie od $3\%$ do $70\%$. W~przypadku rozważanego zadania grupowania zawodników, wartość ta powinna być możliwie jak największa, ponieważ w~zbiorze profesjonalnych zawodników, nie należy spodziewać się sportowców, którzy nie nadają się do gry na jakiejkolwiek pozycji.

\begin{figure}[ht]
    \centering
    \includegraphics[width=1.1\columnwidth]{./figures/dbscan/noise.pdf}
    \caption{Liczba punktów zakwalifikowanych jako szum w~zależności od minimalnej liczby sąsiadów oraz maksymalnego promienia sąsiedztwa}
    \label{fig:dbscan-noise}
\end{figure}

Dla utworzonych etykietowań, możliwym jest wyznaczenie metryk jakości wewnętrznej, podobnie jak w~\ref{sec:wyniki} oraz~\ref{subsec:dalsze-ograniczenie-liczby-atrybutow}. Na podstawie poprzednich wykresów, wybraliśmy $M=5$ jako potencjalnie interesującą wartość parametru. W~celu doboru odpowiedniego $\varepsilon$ obliczyliśmy średnią szerokość sylwetki dla wszystkich etykietowań, która została przedstawiona na rysunku~\ref{fig:dbscan-sil}. Na podstawie wykresu, możemy wyznaczyć rozsądną maksymalną odległość równą $\num{0.375}$.

\begin{figure}[ht]
    \centering
    \includegraphics[width=1.1\columnwidth]{./figures/dbscan/sil.pdf}
    \caption{Porównanie wskaźnika średniej szerokości sylwetki dla grupowania algorytmem DBSCAN}
    \label{fig:dbscan-sil}
\end{figure}

Przedstawiona metoda grupowania gęstościowego ma niewątpliwie swoje wady i~zalety~\cite{kryszkiewicz}. Dla wyznaczonego $M$~oraz~$\varepsilon$, wartość wskaźnika średniej szerokości sylwetki była najwyższa, co wskazuje na potencjalnie najlepiej zdefiniowane i~odseparowane grupy. Niestety, to grupowanie jest niezwykle stratne, ponieważ ponad $80\%$ została zakwalifikowana jako szum i nie została przyporządkowana do grupy.

\section{Podsumowanie}
\label{sec:podsumowanie}

W~ramach projektu, przeprowadzona została analiza problemu automatycznego grupowania zawodników z~gry FIFA~19, według ich nominalnej pozycji. Przetestowano działanie sześciu różnych algorytmów grupowania, przy zastosowaniu czterech różnych miar niepodobieństwa. Dodatkowo, jakość grupowania została zbadana przy użyciu siedmiu różnych wskaźników.

Uzyskane wyniki są dalekie od zadowalających. Dla pierwszego podejścia do grupowania, korzystającego ze~zbioru $20$ atrybutów, obliczone wskaźniki jakości wskazują na niską jakość wewnętrzną uzyskanych grup. W~ogólności, uzyskane etykietowania są słabo określone, rzadkie oraz nieseparowalne. 

Dla grupowań korzystających z~klasycznych metryk niepodobieństwa, dzielących zbiór na dwie grupy, wartości miar zewnętrznych osiągają maksymalne wartości. Algorytmy skutecznie podzieliły zawodników na tych grających w~polu oraz tych grających na bramce. Stało się tak, ponieważ pomiędzy w~analizowanej przestrzeni atrybutów znajdował się jawny atrybut odpowiadający za grę na bramce (którego wartości były wysokie dla bramkarzy a niskie dla reszty zawodników).

Pozostałe rodzaje grupowań ze względu na ogólną i~szczegółową pozycję nie odniosły już takiego sukcesu. Wynika to z~braku jawnego atrybutu opisującego jak dobry zawodnik jest na danej pozycji oraz z~faktu że na jednej pozycji grają zarówno zawodnicy o wysokich i~niskich wartościach atrybutu. W~przypadku grupowania ze względu na pozycję, bardziej interesujące są wzorce jak poszczególne atrybuty mają się do siebie a~nie bezwzględne różnice pomiędzy poszczególnymi wartościami. 

Z~tego powodu, duże nadzieje wiązaliśmy z~grupowaniem opartym o~miarę korelacyjną. Niestety, wyniki przedstawione w~sekcji~\ref{sec:wyniki} ukazały niewątpliwą porażkę tego podejścia. Uzyskane wyniki zdecydowanie odstawały od wyników uzyskanych dla pozostałych miar. Problemy tej miary mogą być związane z~za dużą liczbą atrybutów, które wprowadzały niepotrzebny szum informacyjny.

Dużą poprawę przyniosło ograniczenie przestrzeni atrybutów z~dwudziestu do czterech. Dla tak przygotowanego zbioru, wyniki grupowania zdecydowanie poprawiły się. Dla miar klasycznych, kształt zależności wskaźnika średniej szerokości sylwetki nie zmienił się, jednak wartość bezwzględna wzrosła o~średnio $\num{0.2}$. Dla miary korelacyjnej, zależność diametralnie zmieniła charakter i~utrzymywała stały poziom niezależnie od liczby grup, na względnie wysokim poziomie. 

Grupy uzyskane w~wyniku grupowania danych na ograniczonej przestrzeni atrybutów charakteryzowały się wyższą jakością wewnętrzną: były lepiej zdefiniowane, gęstsze oraz bardziej separowalne. Jakość zewnętrzna grupowania uległa niewielkiej poprawie, szczególnie w~przypadku grupowania \emph{ze względu na ogólną pozycję}. Okazuje się że z~racji mnogości klas, grupowanie ze względu na szczegółową pozycję jest zadaniem wyjątkowo trudnym.

Interesującym elementem naszych badań było przetestowanie algorytmów grupowania rozmytego. Badany zbiór danych opisujących piłkarzy, w~naszej opinii idealnie nadaje się do grupowania rozmytego, ponieważ pozycja na boisku jest w~ogólności zmienną rozmytą. W~przypadku grupowania na pełnym zbiorze atrybutów, wartości rozmytego wskaźnika średniej szerokości sylwetki pozostawiały sporo do życzenia, informując o~niskiej jakości wewnętrznej grupowania. Zmniejszenie liczby atrybutów pozwoliło na zwiększenie wartości wskaźnika, zarówno dla algorytmu FANNY, jak i~algorytmu rozmytych k-średnich. 

Zarówno dla zbioru pełnego, jak i~ograniczonego, najlepsze rezultaty grupowania rozmytego udało nam się uzyskać dla miary korelacyjnej, z~którą wiązaliśmy największe nadzieje na potwierdzenie pierwotnej hipotezy. Najwyższe wyniki wskaźników jakości udało uzyskać się dla wysokich wartości $k$, co może świadczyć że przyjęty podział na szczegółową pozycję jest zbyt wąski. Dokładna analiza uzyskanych grup, powinna umożliwić odkrycie bardziej szczegółowych pozycji, omówionych w~\ref{subsec:atrybuty}.

Zastosowanie grupowania gęstościowego nie pozwoliło nam uzyskać zadowalających wyników. Optymalne rozwiązanie, cechowało się wysoką jakością wewnętrzną, jednak klasyfikowało większość zawodników jako szum, co w~naszej opinii jest zachowaniem błędnym. Mimo to, algorytm DBSCAN jest bardzo interesujący i~nie należy ignorować jego istnienia, uzyskane wyniki mogą być wykorzystane w~innych zadaniach obejmujących analizę zawodników w~profesjonalnych drużynach piłkarskich.

Uzyskane wyniki nie pozwalają nam jednoznacznie potwierdzić lub odrzucić hipotezy postawionej na samym początku. Wyniki uzyskane za pomocą grupowania opartego o~miary klasycznej pokazuje że możliwym jest skuteczne oddzielenie zawodników z~pola od bramkarzy. Wyniki uzyskane przez miarę korelacyjną wskazują że możliwym jest grupowanie po wzorcach, jednak otrzymane wyniki są zdecydowanie za słabe, aby uznać że udało się skutecznie pogrupować zawodników po ich optymalnej pozycji. 

Z~realizacji projektu wyciągnęliśmy dwie nauczki na przyszłość. Pierwszą z~nich jest fakt że \emph{im więcej, tym wcale nie jest lepiej}. Zbyt szeroka przestrzeń atrybutów może wprowadzać za duży szum, co może wpływać na uzyskaną jakość grupowania. Drugą nauczką na przyszłość jest możliwość wykorzystywania miar niepodobieństwa, innych niż klasyczne. W~zadaniach gdzie interesują nas wzorce a~nie wartości bezwzględne, lepiej jest korzystać z~miar opartych o~korelację atrybutów lub innych miar kosinusowych.

\bibliographystyle{abbrv}
\bibliography{bibliography}

\end{document}