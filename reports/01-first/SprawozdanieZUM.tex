\documentclass{article}
\pdfpagewidth=8.5in
\pdfpageheight=11in

\usepackage{../ZUMreport}
\usepackage{times}
\usepackage{url}
\usepackage{xcolor}
\usepackage{polski}
\usepackage[polish]{babel}
\usepackage[utf8]{inputenc}
\usepackage[T1]{fontenc}
\usepackage[utf8]{luainputenc}
\usepackage[hidelinks]{hyperref}
\usepackage[utf8]{inputenc}
\usepackage{caption}
\usepackage{indentfirst}
\usepackage{graphicx}
\usepackage{amsmath}
\usepackage{siunitx}
\usepackage{booktabs}
\usepackage{subfig}

\urlstyle{same}
	
\title{Zaawansowane uczenie maszynowe\\ Założenia wstępne projektu}

\author{
Robert Wojtaś, Jakub Sikora
\affiliations
numery albumów: xxxxxx, 283418 \\
\emails
robert.wojtas.stud@pw.edu.pl, jakub.sikora2.stud@pw.edu.pl
}

\newcommand{\todo}[1]{\textcolor{red}{\textbf{TO DO:} #1}}

\begin{document}
\maketitle

\noindent \textbf{Temat projektu}: Grupowanie (G), \\
\textbf{Zestaw danych}: Dane FIFA 2019 Complete Player Dataset (Kaggle) -- grupowanie lub predykcja skuteczności zawodników.

\section{Cel projektu}
\label{sec:cel-projektu}
Celem projektu będzie zbadanie użyteczności algorytmów grupowania w~zadaniu dobierania optymalnej pozycji piłkarzy z~gry FIFA~19 na boisku. Umiejętności poszczególnych zawodników są opisane za pomocą szeregu parametrów, które mają za zadanie obrazować zdolności gry w~ataku, pomocy i~obronie. Zadaniem algorytmu grupowania będzie połączenie w~grupy zawodników o~podobnych umiejętnościach, tak aby na tej podstawie móc wnioskować o~ich optymalnej pozycji.

\section{Charakterystyka zbioru danych}
\label{sec:dane}
\todo{opisać dane~\cite{fifa-dataset}, jakie są kolumny, co wywalimy a co zostawimy, jakie atrybuty będą obserwowalne a jakie ukryte}

\todo{określić zakres przygotowania danych}

\todo{wskazać możliwości zdefiniowania nowych atrybutów}

\section{Badane algorytmy}
\label{sec:algorytmy}
\todo{ustalić kryteria lub algorytmy selekcji atrybutów obserwowalnych}

\todo{wybrać algorytmy grupowania i wskazać miary \textbf{niepodobieństwa} przykładów}

\todo{wskazać parametry wymagające strojenia}

\section{Miary jakości}
\label{sec:miary-jakosci}
\todo{ustalić procedury i kryteria oceny jakości}

\section{Otwarte kwestie}
\label{sec:otwarte-kwestie}
\todo{do zostawienia sposób opisu grup i charakteryzowania ich specyfiki na podstawie rozkładu wartości atrybutów}

\bibliographystyle{abbrv}
\bibliography{bibliography}

\end{document}